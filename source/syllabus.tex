\documentclass{tufte-book}

% set hyperlink attributes
\hypersetup{colorlinks}

% set table attributes
\usepackage{tabu}
\usepackage{booktabs}

% set bulleted list attributes
\usepackage[inline]{enumitem}

% set table of contents attributes
\setcounter{tocdepth}{2}

% set language attributes
\usepackage[utf8]{inputenc}
\usepackage[T1]{fontenc}

% create full width measurement
\makeatletter
\newlength{\fullwidthlength}
\AtBeginDocument{\setlength{\fullwidthlength}{\@tufte@fullwidth}}
\makeatother

% ============================================================

% define the title
\title{Syllabus}
\author{Christopher G. Prener, Ph.D.}
\publisher{SOC 1120-02: \\ \noindent Introduction to Sociology: Diversity \& Health \\ \noindent \hrulefill \\ \vspace{5mm} \noindent Spring, 2018 \\ \noindent Saint Louis University}

% ============================================================
\begin{document}
% ============================================================
\maketitle % generates the title
% ============================================================
\chapter{Basics}
\section{Course Meeting Times}
Monday \& Wednesday, 2:10 to 3:25 pm
\par \noindent Morrissey Hall 0600

\vspace{6mm}
\section{Course Website}
Web content can be accessed via \textbf{\href{https://classroom.google.com}{Google Classroom}} (\href{https://classroom.google.com}{https://classroom.google.com})

\vspace{6mm}
\section{Contact Information}
\textit{Office:} 1918 Morrissey Hall
\par \noindent \textit{Email:} \href{mailto:chris.prener@slu.edu}{chris.prener@slu.edu}

\vspace{6mm}
\section{Office Hours}
Wednesdays, 10:00am to 12:00pm in 3600 Morrissey (GeoSRI Lab)

\chapter{Course Introduction}
\vspace{2mm}
\begin{fullwidth}
\textit{The function of sociology, as of every science, is to reveal that which is hidden.} \\
\noindent Pierre Bourdieu (1996)
\end{fullwidth}
\vspace{2mm}

\section{Course Description}
This course will survey the field of sociology, stressing important ideas, methods, and results. We focus on health to illustrate the application of sociological ideas. The survey is designed to develop analytic thinking skills. Weekly readings from a text will be supplemented with articles and chapters illustrating topical issues and exercises on the skills and craft of the social sciences.

\vspace{3mm}
\section{Course Objectives}
This course introduces the distinct sociological skills through the lens of health and illness, including:
\begin{enumerate}[leftmargin=!,labelindent=5pt]
\itemsep-.25em
\item The ability to recognize and examine social phenomena from multiple perspectives.
\item The recognition of what constitutes fact based arguments from appropriately designed information gathering.
\item The ability to understand the sources of attitudes and behaviors from cultures and structures and how they impact the quality of life of different groups in society.
\item The ability to reflect on the diversity around us and to act in a moral and just manner as citizens of the world.
\item Developing skills in independent thinking, aesthetic awareness, moral and/or ethical system of values, welcoming diversity, and committing to the value of life-long learning.
\end{enumerate}

\vspace{3mm}
\section{Cultural Diversity Core Requirement}
This course fulfills the College of Arts and Sciences core requirement for Cultural Diversity in the United States. The Cultural Diversity in the United States requirement is designed to help students gain a better understanding of the cultural groups in the United States and their interactions. Students who complete a Cultural Diversity course in this category will gain a substantial subset of the following skills:
\begin{enumerate}[leftmargin=!,labelindent=5pt]
\itemsep-.25em
\item Analyze and evaluate how various underrepresented social groups confront inequality and claim a just place in society.
\item Examine how conflict and cooperation between social groups shapes U.S. society and culture.
\item Identify how individual and institutional forms of discrimination impact leaders, communities and community building through the examination of such factors as race, ethnicity, gender, religion, economic class, age, physical and mental capability, and sexual orientation.
\item Evaluate how their personal life experiences and choices fit within the larger mosaic of U.S. society by confronting and critically analyzing their own values and assumptions about individuals and groups from different cultural contexts.
\item Understand how questions of diversity intersect with moral and political questions of justice and equality.
\end{enumerate}

\vspace{3mm}
\section{Google Classroom}
\textbf{\href{https://classroom.google.com}{Google Classroom}} is a learning management system like Blackboard, which you may be more familiar with. There are two main areas - the `Stream' and the `About' tabs. The `Stream' contains posts for announcements and assignments. Additions to the `Stream' should be emailed to your student e-mail account automatically. This will be my primary means for communicating with the class as a whole. Assignments posted to the `Stream' allow you to submit work for the course. You can use the `Topics' on the left-hand menu of the `Stream' to filter posts.

\vspace{3mm}
\section{Core Documents}
There are two core documents for this course. This \textbf{Syllabus} sets out core expectations and policies for the course. The \textbf{Reading List} contains topics, required readings, and assignment due dates for each week. These two documents spell out what is \textit{required} for this course. Both documents are available under the `About' tab on \textbf{\href{https://classroom.google.com}{Google Classroom}}. 

\vspace{3mm}
\section{Readings}
There are two books required for this course. Each book has been selected to correspond with one or more of the course objectives. The books are:
\begin{enumerate}[leftmargin=!,labelindent=5pt,itemindent=-15pt]
\itemsep-.25em
\item Abraham, Laurie K. 1993. \textit{Mama Might Be Better Off Dead: The Failure of Health Care in Urban America}. Chicago, IL: The \\ University of Chicago Press. ISBN-13: 978-0226001395; \\ List Price: \$20.00; e-book versions available.
\item Andersen, Margaret, Howard F. Taylor, and Kim A. Logio. 2016. \textit{Sociology: The Essentials}. 9\textsuperscript{th} edition. Independence, KY: Cengage. ISBN-13: 978-1305503083; List Price: \$202.95; e-book versions available.\sidenote{Older versions of this book are available. All course content and exam questions are based on the content available in the 9\textsuperscript{th} edition, and I cannot guarantee that older editions will cover all of material discussed in class. Students who are concerned about the price are instead encouraged to explore rental or e-book options.}
\end{enumerate}

\par I do not require students to buy physical copies of texts. You are free to select a means for accessing these texts that meets your budget and learning style. If ebook editions (e.g. Kindle, iBooks, pdf, etc) of texts are available, they are acceptable for this course. All texts should be obtained in the edition noted above. 

\par All readings are listed on the \textbf{Reading List} and should be completed before the course meeting on the week in which they are assigned (unless otherwise noted). Full text versions of most readings not found in the books assigned for the course can be obtained using the library's \href{http://eres.slu.edu/eres/coursepass.aspx?cid=4443}{Electronic Reserves} system. The password for the Electric Reserves will be posted on \textbf{\href{https://classroom.google.com}{Google Classroom}}.

% ============================================================
\chapter{Course Expectations and Policies}
\section{General Expectations for Students}
My priority is that class periods are productive learning experiences for all students. In order to foster this type of productive environment, I ask students to follow a few policies:
\begin{enumerate}[leftmargin=!,labelindent=5pt]
\itemsep-.25em
	\item Arrive to class on time and stay for the entire class period.\sidenote{If you drive to campus, please get an on-campus parking pass for the semester or use a smartphone app to top off your meter. Leaving class to feed your meter is disruptive for both you and your classmates.}
	\item Silence \textit{all} electronic devices before entering the classroom.
	\item Do not engage in side conversations. This is disrespectful to the speaker (whether me or a classmate), and can affect the ability of others in the class to learn.
	\item Be respectful of your fellow classmates. Do not interrupt when someone is speaking, monopolize the conversation, or belittle the ideas or opinions of others. 
	\item Complete the assigned readings for each class in advance, and come prepared with discussion points and questions. 
\end{enumerate}

\vspace{3mm}
\section{Course Content}
This class covers topics that have a long history of divisiveness in the United States. Students can expect to confront viewpoints that they may disagree with or find contentious. I do not expect that students will always agree with each other (or with me), and I encourage healthy discussion and debate in class. However, I will insist that all such conversations be conducted respectfully. 

\vspace{3mm}
\section{Attendance and Participation}
Attendance and participation are important components of this course. Students are expected to attend all class sessions. Making up missed classes are your responsibility, including obtaining notes from a classmate. I do post slides on \textbf{\href{https://classroom.google.com}{Google Classroom}}, but my slides are intended only to serve as references. All lecture slides will be posted the week after their use in class. Please note that lectures and discussions cannot be recorded by any means (e.g. audio or video recordings, or photographs) without my permission.

\vspace{3mm}
\section{Communication}
Email is my preferred method of communication. I dedicate time to email responses each workday, meaning that my response time is typically within 24 hours during the workweek. If you have not received a response from me after 48 hours (or by end of business on Monday if you emailed me over the weekend), please follow-up to ensure that your message did not get lost in the shuffle. 

\par Please use your SLU email account when emailing me. All messages regarding course updates, assignments, and changes to the class schedule including cancellations will be sent to your SLU email account. It is imperative that you check your SLU email account regularly. 

\par Please also ensure that all concerns or questions about your standing in the course are directed to me immediately. Inquires from parents, SLU staff members, and others will not be honored.

\vspace{3mm}
\section{Electronic Devices}
During class periods, students are asked to refrain from using electronic devices (including cell phones) for activities not directly related to the course. For this class, I expect students to limit their use of electronic devices to accessing course readings, notes, and other course materials.

\par There is evidence that using electronic devices during lectures results in decreased retention of course content \sidenote[][-26mm]{Hembrooke, Helene and Geri Gay. 2003. ``The Laptop and the Lecture: The Effects of Multitasking in Learning Environments''. \textit{Journal of Computing in Higher Education} 5(1): 46-64.} and lower overall course performance.\sidenote[][-10mm]{Fried, Carrie. 2008. ``In-class laptop use and its effects on student learning''. \textit{Computers \& Education} 50(3): 906-914.}  Students who are not using a laptop but are in direct view of another student's laptop also have decreased performance in courses.\sidenote[][-6mm]{Sana, Faria et al. 2013. ``Laptop Multitasking Hinders Classroom Learning for Both Users and Nearby Peers''. \textit{Computers \& Education} 62: 24-31.} Conversely, students who take notes the ``old fashioned way'' have better performance on tests compared to students who take notes on laptops.\sidenote[][2mm]{Mueller, Pam and Daniel Oppenheimer. 2014. ``The Pen Is Mightier Than the Keyboard Advantages of Longhand Over Laptop Note Taking''. \textit{Psychological Science} 25(6): 1159-1168.} 

\par I therefore ask students to be conscious of how they are using their devices, the ways such use impacts their own learning, and the effect that it may have on others around them. I reserve the right to alter this policy if electronic device use becomes problematic during the semester.

\vspace{3mm}
\section{Student Support}
If you meet the eligibility requirements for \textbf{academic accommodations} through the Disability Services office (located within the Student Success Center), you should arrange to discuss your needs with me after the first class. All discussions of this nature are treated confidentially, and I will make every effort to work with you to come up with a plan for successfully completing the course requirements. Please note that I will not provide accommodations to students who are not working with Disability Services.\sidenote[][-13mm]{Additional details can be found on the Disability Services \href{http://www.slu.edu/retention-and-academic-success/disability-services}{website}. You can contact them at \\ \noindent \href{mailto:disability\_services@slu.edu}{disability\_services@slu.edu} or \\ \noindent 314-977-3484 to schedule an appointment.}

\par If you are a \textbf{student-athlete} who is in-season, you should discuss your game schedule with me after the first class and share your travel letter with me as soon as you have a copy. You are reminded that games and tournaments are not excuses for failing to complete assignments, and that NCAA rules prohibit student-athletes from missing classes for practice. Low grades that jeopardize eligibility must be addressed immediately by you, not by a coach or academic coordinator.\sidenote{More information about resources and academic support for student-athletes can be found at the Student-Athlete Academic Support Services \href{http://www.slubillikens.com/ViewArticle.dbml?&DB_OEM_ID=27200&ATCLID=205243370}{website}.}

\par I also encourage you to take advantage of the \textbf{University Writing Services (UWS) program}. Getting feedback benefits writers at all skill levels and the quality of your writing will be reflected in assignment grades. The UWS has trained writing consultants who can help you improve the quality of your written work. UWS's consultants are available to address everything from brainstorming and developing ideas to crafting strong sentences and documenting sources.\sidenote[][-10mm]{More information on the UWS program can be found on their \href{http://www.slu.edu/retention-and-academic-success/university-writing-services}{website}. The UWS program has a number of on-campus locations.}

\vspace{3mm}
\section{Academic Honesty}
All students should familiarize themselves with \href{http://www.slu.edu/Documents/provost/academic_affairs/Academic\%20Integrity\%20Policy\%20FINAL\%20\%206-26-15.pdf}{Saint Louis University's policies} concerning cheating, plagiarism, and other academically dishonest practices:\sidenote[][-10mm]{This course is also governed by the College of Arts and Sciences' academic honesty policies, which are available on their \href{http://www.slu.edu/college-of-arts-and-sciences-home/undergraduate-education/academic-honesty}{website}.} 

\vspace{3mm}
\begin{adjustwidth}{1cm}{}
\textit{Academic integrity is honest, truthful and responsible conduct in all academic endeavors. The mission of Saint Louis University is ``the pursuit of truth for the greater glory of God and for the service of humanity.''  Accordingly, all acts of falsehood demean and compromise the corporate endeavors of teaching, research, health care, and community service via which SLU embodies its mission. The University strives to prepare students for lives of personal and professional integrity, and therefore regards all breaches of academic integrity as matters of serious concern.}
\end{adjustwidth}

\vspace{3mm}
\noindent Any work that is taken from another student, copied from printed material, or copied the internet without proper citation is expressly prohibited. All relevant assignments should include in-text citations and references formatted using the \href{http://owl.english.purdue.edu/owl/resource/583/02/}{American Sociological Association (ASA)} or the \href{https://owl.english.purdue.edu/owl/resource/560/01/}{American Psychological Association (APA)} style guidelines. Any student who is found to have been academically dishonest in their work risks failing both the assignment and this course.

\vspace{3mm}
\section{Title IX}
All students should familiarize themselves with \href{http://www.slu.edu/general-counsel-home/office-of-institutional-equity-and-diversity}{Saint Louis University's polices} on bias, discrimination, and harassment:

\vspace{3mm}
\begin{adjustwidth}{1cm}{}
\textit{Saint Louis University and its faculty are committed to supporting our students and seeking an environment that is free of bias, discrimination, and harassment. If you have encountered any form of sexual misconduct (e.g. sexual assault, sexual harassment, stalking, domestic or dating violence), we encourage you to report this to the University. If you speak with a faculty member about an incident of misconduct, that faculty member must notify SLU's Title IX Coordinator, Anna R. Kratky (DuBourg Hall, Room 36; \href{mailto:akratky@slu.edu}{akratky@slu.edu}); 314-977-3886) and share the basic facts of your experience with her. The Title IX coordinator will then be available to assist you in understanding all of your options and in connecting you with all possible resources on and off campus.}
\vspace{2mm} \par \noindent \textit{If you wish to speak with a confidential source, you may contact the counselors at the University Counseling Center at 314-977-TALK.}
\end{adjustwidth}

\vspace{3mm}
\noindent Consistent with the above policy, I will forward all reports of inappropriate conduct to the Title IX Coordinator's office. Please also be aware that communications over various online systems, including (but not limited to) \textbf{\href{https://classroom.google.com}{Google Classroom}} and Google Apps, are also covered by this policy.

% ============================================================
\chapter{Course Assignments and Grading}
\section{Assignments}
\paragraph{Attendance and Participation (10\%)} 
As discussed above, both attendance and participation are important aspects of this class. The class participation grade will be based on (a) attendance, (b) level of engagement during class, and (c) class ``entry'' and ``exit tickets''. 

\par Attendance check-ins will be collected either in-person or through a simple web-form. Students will need a \href{https://en.wikipedia.org/wiki/QR_code}{QR code} reader application installed on their smartphone to check-in. If you do not own a smartphone, you will be able to sign-in with Chris before class. Students should note that attendance check-ins are covered by the course's Academic Honesty policy. Sharing the check-in form with another student or signing in on their behalf are both violations of this policy.

\par ``Entry'' and ``exit'' tickets will be collected via \textbf{\href{https://classroom.google.com}{Google Classroom}}. These will only be graded for students who present in class on the day that they were collected. Students without access to a smartphone or laptop should submit their tickets as soon as possible after class on the days the tickets are collected.

\paragraph{Sociological Experiences (10\%)}
Students will be required to complete two sociological experiences and write a short response paper summarizing your own experience. Additional details and a grading rubric are available on \textbf{\href{https://classroom.google.com}{Google Classroom's}} `Stream' under the `Assignments' topic. ``The Bus'' will be due on \textbf{February 12\textsuperscript{th}} and ``The Suitcase'' will be due on \textbf{March 28\textsuperscript{th}}. Each reflection paper is worth 5\% of your final grade.

\paragraph{QHQ Papers (20\%)} 
Each student will write a QHQ reflection paper on three chapters (one chapter per discussion period) of \textit{Mama Might Be Better Off Dead} (Abraham 1993). These reflection papers will integrate previous lecture material and readings to understand the cycle of events described in the book. Additional details and a grading rubric are available on \textbf{\href{https://classroom.google.com}{Google Classroom's}} `Stream' under the `Assignments' topic. QHQ Paper 1 will be due on \textbf{February 28\textsuperscript{th}}, QHQ Paper 2 will be due on \textbf{April 11\textsuperscript{th}}, and QHQ Paper 3 will be due on \textbf{May 2\textsuperscript{nd}}. QHQ Paper 1 is ungraded. If it is not turned in or not taking seriously, however, a deduction of 3\% will be applied to your final grade. QHQ Papers 2 and 3 each count for 10\% of your final grade.

\paragraph{Exams (60\%)}
Three non-cumulative multiple choice exams will be given throughout the semester. Each exam will cover the breadth of the material in the course, including readings, lectures, and videos. Exam 1 will be given on \textbf{February 19\textsuperscript{th}}, Exam 2 will be given on \textbf{April 4\textsuperscript{th}}, and Exam 3 will be given during our class’s schedule final exam period on \textbf{May 9\textsuperscript{th}} from 2:00pm to 3:50pm. Each exam is worth 20\% of your final grade.

\vspace{3mm}
\section{Submission and Late Work}
\par All assignments must be submitted via \textbf{\href{https://classroom.google.com}{Google Classroom}} \textit{as a Google Doc file} unless otherwise noted in the assignment instructions. \textbf{\href{https://classroom.google.com}{Google Classroom}} will have submission portals for each assignment in the `Stream' under the `Assignments' topic. Create your file in Google Docs and then use the submission portal to submit the file.\sidenote{If you would prefer to create your document offline using Microsoft Word or something similar, copy and paste your completed draft into a new Google Doc when you are ready, ensure that the document is properly formatted, and submit the Google Doc file.} Once submitted, you will not be able to edit the file again until it is returned with feedback and a grade. Feedback will be returned to students via comments embedded in each Google Doc.

\par The Google Doc submission policy is in place because it facilitates clear, easy grading that can be turned around to you quickly. Submitting assignments in ways that deviate from this policy will result in a late grade (see below) being applied in the first instance and a zero grade for each subsequent instance.

\par Once the class begins, any assignments shared will be treated as late. Assignments shared within 24-hours of the beginning of class will have 15\% deducted from the grade. I will deduct 15\% per day for the next two 24-hour periods that assignments are late; after 72-hours, I will not accept late work.\sidenote{These periods begin at 2:10pm each day. All assignments must be turned in by 2:10pm on the third day after they are due to receive partial credit.} If you cannot attend class because of personal illness, a family issue, jury duty, an athletic match, or a religious observance, you must contact me beforehand to discuss alternate submission of work. Internet or computer issues are not grounds for missed deadlines.

\vspace{3mm}
\section{Extra Credit}
From time to time I may offer extra credit to be applied to your final grade. I will only offer extra credit if it is open to the entire class (typically for something like attending a lecture or event on-campus). If I offer extra credit, I will typically require you to submit a short written summary of the activity within a week of the event to obtain the credit. Papers should be submitted via \textbf{\href{https://classroom.google.com}{Google Classroom}}. When offered, extra credit opportunities cannot be made-up or substituted if you are unable to attend the event.

\vspace{3mm}
\section{Grading}
Grades will be included with assignment feedback and will also be made available through Blackboard's grade center.\sidenote{This is the only official use of Blackboard for this course.} I use a point system for calculating grades. The following table gives the weighting and final point totals for all assignments for this course: \\
\vspace{3mm}
\noindent
\begin{tabu} to 1\fullwidthlength { X[l] X[.4cm,r] X[.5cm,r] X[.3cm,r] X[.5cm,r] }
\multicolumn{5}{c}{Grading Point System}\\
\multicolumn{1}{c}{Assignment} & \multicolumn{1}{c}{Weight} & \multicolumn{1}{c}{Points} & \multicolumn{1}{c}{Quantity} & \multicolumn{1}{c}{Total Points} \\
 \hline
 \hline
 Attendance and Participation & 10\% & 50 points & x1 & 50 points \\
 Sociological Experiences & 10\% & 25 points & x2 & 50 points \\
 QHQ Papers & 20\% & 50 points & x2 & 100 points \\
 Exams & 60\% & 100 points & x3 & 300 points \\
 \hline
 \textit{Total} & 100\% & & & 500 points \\
 \hline   
\end{tabu}
\vspace{3mm}

\par I will return assignments with grades that represent number of points earned. If you want to know your percentage on a particular assignment, divide the number of points earned by the number of points possible and then multiply it by 100.

\par Final grades will be calculated by taking the sum of all points earned and dividing it by the total number of points possible (1,000). This will be multiplied by 100 and then converted to a letter grade using the following table:\\
\vspace{3mm}
\noindent 
\begin{tabu} to 1\textwidth { X[l] X[l] X[l] }
 \multicolumn{3}{c}{Final Grading Scale}\\
 \multicolumn{1}{c}{Grade Point} & \multicolumn{1}{c}{Letter} & \multicolumn{1}{c}{Percentage} \\
 \hline
 \hline
 4.0 & A & 93.0-100 \\
 3.7 & A- & 90.0-92.9 \\
 3.3 & B+ & 87.0-89.9 \\
 3.0 & B & 83.0-86.9 \\
 2.7 & B- & 80.0-82.9 \\
 2.3 & C+ & 77.0-79.9 \\
 2.0 & C & 73.0-76.9 \\
 1.7 & C- & 70.0-72.9 \\
 1.0 & D & 63.0-69.9 \\
 0.0 & F & < 63.0 \\
 \hline
\end{tabu}
\vspace{3mm}

\noindent Incomplete grades will be given upon request only if you have a ``C'' average and have completed at least two-thirds of the assignments. You should note that incomplete grades must be rectified by the specified deadline or they convert to an ``F''.
\newpage

% ============================================================
\chapter{Course Schedule}
\section{Course Overview}
\begin{tabu} to 1\textwidth { X[.1cm,c] X[.6cm,l] X{l} }
Week & \multicolumn{1}{c}{Monday Date} &  \multicolumn{1}{c}{Topics} \\
\hline
\hline
1 & January 15\textsuperscript{th} & Course Introduction \\
2 & January 22\textsuperscript{nd} & Social Theory and Health \\
3 & January 29\textsuperscript{th} & Culture \\
4 & February 5\textsuperscript{th} & Socialization \\
5 & February 12\textsuperscript{th} & Social Structure \\
6 & February 19\textsuperscript{th} &  Exam 1 \& Urban Sociology \\
7 & February 26\textsuperscript{th} & Urban Sociology \& \textit{Mama}, Part 1 \\
8 & March 5\textsuperscript{th} & Crime \& Deviance \\
9 & March 12\textsuperscript{th} & \textit{No Class: Spring Break} \\
10 & March 19\textsuperscript{th} & Class \& Stratification \\
11 & March 26\textsuperscript{th} & Race \& Ethnicity \\
12 & April 2\textsuperscript{nd} & \textit{No Class: Easter Break} \& Exam 2 \\
13 & April 9\textsuperscript{th} & Gender \& \textit{Mama}, Part 2 \\
14 & April 16\textsuperscript{th} & The Illness Experience \\
15 & April 23\textsuperscript{rd} & Economy \& Politics \\
16 & April 30\textsuperscript{th} & Native American Health \& \textit{Mama}, Part 3 \\
17 & May 7\textsuperscript{th} & Course Conclusion \& Exam 3 \\
\hline
\end{tabu}

\vspace{3mm}
\section{Reading List}
Please consult the stand-alone \textbf{Reading List} document for details on readings and assignments for each week.\sidenote{All course documents including the \textbf{Reading List} are available under the `About' tab on \textbf{\href{https://classroom.google.com}{Google Classroom}}.}

\vspace{3mm}
\section{Scheduling Notes}
In the event of a cancellation due to weather or another disruption, I reserve the right to alter the weekly schedule or assignment deadlines. 

% ============================================================
\end{document}