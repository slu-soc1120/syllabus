% Options for packages loaded elsewhere
\PassOptionsToPackage{unicode}{hyperref}
\PassOptionsToPackage{hyphens}{url}
%
\documentclass[
]{book}
\usepackage{amsmath,amssymb}
\usepackage{lmodern}
\usepackage{iftex}
\ifPDFTeX
  \usepackage[T1]{fontenc}
  \usepackage[utf8]{inputenc}
  \usepackage{textcomp} % provide euro and other symbols
\else % if luatex or xetex
  \usepackage{unicode-math}
  \defaultfontfeatures{Scale=MatchLowercase}
  \defaultfontfeatures[\rmfamily]{Ligatures=TeX,Scale=1}
\fi
% Use upquote if available, for straight quotes in verbatim environments
\IfFileExists{upquote.sty}{\usepackage{upquote}}{}
\IfFileExists{microtype.sty}{% use microtype if available
  \usepackage[]{microtype}
  \UseMicrotypeSet[protrusion]{basicmath} % disable protrusion for tt fonts
}{}
\makeatletter
\@ifundefined{KOMAClassName}{% if non-KOMA class
  \IfFileExists{parskip.sty}{%
    \usepackage{parskip}
  }{% else
    \setlength{\parindent}{0pt}
    \setlength{\parskip}{6pt plus 2pt minus 1pt}}
}{% if KOMA class
  \KOMAoptions{parskip=half}}
\makeatother
\usepackage{xcolor}
\IfFileExists{xurl.sty}{\usepackage{xurl}}{} % add URL line breaks if available
\IfFileExists{bookmark.sty}{\usepackage{bookmark}}{\usepackage{hyperref}}
\hypersetup{
  pdftitle={SOC 1120-05 and H05: Introduction to Sociology - Diversity \& Health},
  pdfauthor={Christopher Prener, Ph.D.},
  hidelinks,
  pdfcreator={LaTeX via pandoc}}
\urlstyle{same} % disable monospaced font for URLs
\usepackage{longtable,booktabs,array}
\usepackage{calc} % for calculating minipage widths
% Correct order of tables after \paragraph or \subparagraph
\usepackage{etoolbox}
\makeatletter
\patchcmd\longtable{\par}{\if@noskipsec\mbox{}\fi\par}{}{}
\makeatother
% Allow footnotes in longtable head/foot
\IfFileExists{footnotehyper.sty}{\usepackage{footnotehyper}}{\usepackage{footnote}}
\makesavenoteenv{longtable}
\usepackage{graphicx}
\makeatletter
\def\maxwidth{\ifdim\Gin@nat@width>\linewidth\linewidth\else\Gin@nat@width\fi}
\def\maxheight{\ifdim\Gin@nat@height>\textheight\textheight\else\Gin@nat@height\fi}
\makeatother
% Scale images if necessary, so that they will not overflow the page
% margins by default, and it is still possible to overwrite the defaults
% using explicit options in \includegraphics[width, height, ...]{}
\setkeys{Gin}{width=\maxwidth,height=\maxheight,keepaspectratio}
% Set default figure placement to htbp
\makeatletter
\def\fps@figure{htbp}
\makeatother
\setlength{\emergencystretch}{3em} % prevent overfull lines
\providecommand{\tightlist}{%
  \setlength{\itemsep}{0pt}\setlength{\parskip}{0pt}}
\setcounter{secnumdepth}{5}
\usepackage{booktabs}
\usepackage{amsthm}
\makeatletter
\def\thm@space@setup{%
  \thm@preskip=8pt plus 2pt minus 4pt
  \thm@postskip=\thm@preskip
}
\makeatother

% work around for errors related to the undefined shaded* enviornment:
\usepackage{color}
\usepackage{fancyvrb}
\newcommand{\VerbBar}{|}
\newcommand{\VERB}{\Verb[commandchars=\\\{\}]}
\DefineVerbatimEnvironment{Highlighting}{Verbatim}{commandchars=\\\{\}}
% Add ',fontsize=\small' for more characters per line
\usepackage{framed}
\definecolor{shadecolor}{RGB}{248,248,248}
\newenvironment{Shaded}{\begin{snugshade}}{\end{snugshade}}
\newcommand{\KeywordTok}[1]{\textcolor[rgb]{0.13,0.29,0.53}{\textbf{#1}}}
\newcommand{\DataTypeTok}[1]{\textcolor[rgb]{0.13,0.29,0.53}{#1}}
\newcommand{\DecValTok}[1]{\textcolor[rgb]{0.00,0.00,0.81}{#1}}
\newcommand{\BaseNTok}[1]{\textcolor[rgb]{0.00,0.00,0.81}{#1}}
\newcommand{\FloatTok}[1]{\textcolor[rgb]{0.00,0.00,0.81}{#1}}
\newcommand{\ConstantTok}[1]{\textcolor[rgb]{0.00,0.00,0.00}{#1}}
\newcommand{\CharTok}[1]{\textcolor[rgb]{0.31,0.60,0.02}{#1}}
\newcommand{\SpecialCharTok}[1]{\textcolor[rgb]{0.00,0.00,0.00}{#1}}
\newcommand{\StringTok}[1]{\textcolor[rgb]{0.31,0.60,0.02}{#1}}
\newcommand{\VerbatimStringTok}[1]{\textcolor[rgb]{0.31,0.60,0.02}{#1}}
\newcommand{\SpecialStringTok}[1]{\textcolor[rgb]{0.31,0.60,0.02}{#1}}
\newcommand{\ImportTok}[1]{#1}
\newcommand{\CommentTok}[1]{\textcolor[rgb]{0.56,0.35,0.01}{\textit{#1}}}
\newcommand{\DocumentationTok}[1]{\textcolor[rgb]{0.56,0.35,0.01}{\textbf{\textit{#1}}}}
\newcommand{\AnnotationTok}[1]{\textcolor[rgb]{0.56,0.35,0.01}{\textbf{\textit{#1}}}}
\newcommand{\CommentVarTok}[1]{\textcolor[rgb]{0.56,0.35,0.01}{\textbf{\textit{#1}}}}
\newcommand{\OtherTok}[1]{\textcolor[rgb]{0.56,0.35,0.01}{#1}}
\newcommand{\FunctionTok}[1]{\textcolor[rgb]{0.00,0.00,0.00}{#1}}
\newcommand{\VariableTok}[1]{\textcolor[rgb]{0.00,0.00,0.00}{#1}}
\newcommand{\ControlFlowTok}[1]{\textcolor[rgb]{0.13,0.29,0.53}{\textbf{#1}}}
\newcommand{\OperatorTok}[1]{\textcolor[rgb]{0.81,0.36,0.00}{\textbf{#1}}}
\newcommand{\BuiltInTok}[1]{#1}
\newcommand{\ExtensionTok}[1]{#1}
\newcommand{\PreprocessorTok}[1]{\textcolor[rgb]{0.56,0.35,0.01}{\textit{#1}}}
\newcommand{\AttributeTok}[1]{\textcolor[rgb]{0.77,0.63,0.00}{#1}}
\newcommand{\RegionMarkerTok}[1]{#1}
\newcommand{\InformationTok}[1]{\textcolor[rgb]{0.56,0.35,0.01}{\textbf{\textit{#1}}}}
\newcommand{\WarningTok}[1]{\textcolor[rgb]{0.56,0.35,0.01}{\textbf{\textit{#1}}}}
\newcommand{\AlertTok}[1]{\textcolor[rgb]{0.94,0.16,0.16}{#1}}
\newcommand{\ErrorTok}[1]{\textcolor[rgb]{0.64,0.00,0.00}{\textbf{#1}}}
\newcommand{\NormalTok}[1]{#1}

% create callout boxes:
\newenvironment{rmdblock}[1]
  {\begin{shaded*}
  \begin{itemize}
  \renewcommand{\labelitemi}{
    \raisebox{-.7\height}[0pt][0pt]{
      {\setkeys{Gin}{width=3em,keepaspectratio}\includegraphics{images/#1}}
    }
  }
  \item
  }
  {
  \end{itemize}
  \end{shaded*}
  }
\newenvironment{rmdnote}
  {\begin{rmdblock}{note}}
  {\end{rmdblock}}
\newenvironment{rmdtip}
  {\begin{rmdblock}{tip}}
  {\end{rmdblock}}
\newenvironment{rmdwarning}
  {\begin{rmdblock}{warning}}
  {\end{rmdblock}}

% set part and section names:
\usepackage{fancyhdr}
\renewcommand{\chaptername}{Section}
\renewcommand\thesection{\Alph{section}}
\ifLuaTeX
  \usepackage{selnolig}  % disable illegal ligatures
\fi
\usepackage[]{natbib}
\bibliographystyle{apalike}

\title{SOC 1120-05 and H05: Introduction to Sociology - Diversity \& Health}
\author{Christopher Prener, Ph.D.}
\date{2021-08-20}

\begin{document}
\maketitle

\begin{center}
{\huge Preface and Warning} \\
\end{center}
\vspace{5mm}
This is the hardcopy version of the \textbf{Fall 2021} syllabus.
\vspace{5mm}
\par \noindent This \texttt{.pdf} version of the course syllabus is automatically created as part of the document generation process. It is meant for students who wish to keep a hardcopy of the course policies and planned course schedule. \textbf{Since it is automatically created, it is not optimized for easy use} - readers may notice formatting inconsitencies and stray characters that are a result of the markdown to \LaTeX{} conversion process. The web version (located at \href{https://slu-soc1120.github/syllabus/}{https://slu-soc1120.github/syllabus/}) is meant to be the version of the syllabus used for everyday reference during the semester. As such, this \texttt{.pdf} version will not be updated as the semester progresses should any changes to the course schedule be necessary.

\hypertarget{basics}{%
\chapter*{Basics}\label{basics}}
\addcontentsline{toc}{chapter}{Basics}

\hypertarget{course-meetings}{%
\subsection*{Course Meetings}\label{course-meetings}}
\addcontentsline{toc}{subsection}{Course Meetings}

\emph{When:} Tuesday and Thursday, 11:00am CST to 12:15pm CST

\emph{Where:} Morrissey Hall 0200

\hypertarget{course-website}{%
\subsection*{Course Website}\label{course-website}}
\addcontentsline{toc}{subsection}{Course Website}

\url{https://slu-soc1120.github.io}

\hypertarget{course-materials}{%
\subsection*{Course Materials}\label{course-materials}}
\addcontentsline{toc}{subsection}{Course Materials}

\url{https://blackboard.slu.edu}

\hypertarget{chriss-information}{%
\subsection*{Chris's Information}\label{chriss-information}}
\addcontentsline{toc}{subsection}{Chris's Information}

\begin{rmdwarning}
Please note that while I am teaching face to face this semester, all
other interactions for this course will take place virtually.
\end{rmdwarning}

\textbf{Email:} \href{mailto:chris.prener@slu.edu}{\nolinkurl{chris.prener@slu.edu}}

\textbf{Office Hours, Appointment Only:} Wednesdays, 9:00 AM CST to 10:30 AM CST; sign-up via Calendly to receive personalized calendar and Zoom invitations (SLU log-in required)

\hypertarget{hard-copy-syllabus}{%
\section*{Hard-copy Syllabus}\label{hard-copy-syllabus}}
\addcontentsline{toc}{section}{Hard-copy Syllabus}

If you would like to keep a record of the syllabus, there is a \texttt{.pdf} download button () in the top toolbar.This document will contain a ``snapshot'' of the course policies and planned schedule as of the beginning of the semester but will not be subsequently updated. See the ``Preface and Warning'' on page 2 of the \texttt{.pdf} for additional details.

\hypertarget{change-log}{%
\section*{Change Log}\label{change-log}}
\addcontentsline{toc}{section}{Change Log}

\begin{itemize}
\tightlist
\item
  August 20, 2021 - Add schedule for Fall 2021
\end{itemize}

\hypertarget{license}{%
\section*{License}\label{license}}
\addcontentsline{toc}{section}{License}

Copyright © 2016-2021 \href{https://chris-prener.github.io}{Christopher G. Prener}

This work is licensed under a Creative Commons Attribution-ShareAlike 4.0 International License.

\hypertarget{part-syllabus}{%
\part{Syllabus}\label{part-syllabus}}

\hypertarget{course-introduction}{%
\chapter{Course Introduction}\label{course-introduction}}

\begin{quote}
The function of sociology, as of every science, is to reveal that which is hidden.
\end{quote}

\textbf{Pierre Bourdieu (1996)}

This course will survey the field of sociology, stressing important ideas, methods, and results. We focus on health to illustrate the application of sociological ideas. The survey is designed to develop analytic thinking skills. Weekly readings from a text will be supplemented with articles and chapters illustrating topical issues and exercises on the skills and craft of the social sciences.

\hypertarget{two-courses-one-goal}{%
\section{Two Courses, One Goal}\label{two-courses-one-goal}}

Students will quickly notice that this course has two numbers. SOC 1120-05 is the ``regular'' course section, and SOC 1120-05H is the honors section. If you are a \href{https://www.slu.edu/honors/index.php}{University Honors} student, you may enroll in SOC 1120-05H if you wish to take this course for honors credit. Students in the honors section complete additional readings for three of the weeks, write several additional response papers, and give a presentation on a topic of their choice to the class. Additional details are include in the \href{/syllabus/honors-overview.html}{honors supplement at the end of this document}. Both courses have the same goal - to introduce students to the fundamentals of sociology through an emphasis on health and medicine.

\hypertarget{course-objectives}{%
\section{Course Objectives}\label{course-objectives}}

By the \emph{end} of the semester, you should be able to:

\begin{enumerate}
\def\labelenumi{\arabic{enumi}.}
\tightlist
\item
  Describe the major theoretical traditions within sociology and the way that we use social theory, and apply these theories to current events.
\item
  Identify sociological contributions to a number of substantive areas, including urban sociology, crime and deviance, race, class, and gender.
\item
  Apply core sociological concepts by analyzing data and your own experiences to understand how they reflect fundamental social issues.
\item
  Integrate core sociological concepts into analyses of population health and health disparities using both fundamental cause theory and the social determinants of health perspective as well as other sociological concepts and data.
\end{enumerate}

\hypertarget{cultural-diversity-core-requirement}{%
\section{Cultural Diversity Core Requirement}\label{cultural-diversity-core-requirement}}

This course fulfills the College of Arts and Sciences core requirement for Cultural Diversity in the United States. The Cultural Diversity in the United States requirement is designed to help students gain a better understanding of the cultural groups in the United States and their interactions. Students who complete a Cultural Diversity course in this category will gain a substantial subset of the following skills:

\begin{enumerate}
\def\labelenumi{\arabic{enumi}.}
\item
  Analyze and evaluate how various underrepresented social groups confront inequality and claim a just place in society.
\item
  Examine how conflict and cooperation between social groups shapes U.S. society and culture.
\item
  Identify how individual and institutional forms of discrimination impact leaders, communities and community building through the examination of such factors as race, ethnicity, gender, religion, economic class, age, physical and mental capability, and sexual orientation.
\item
  Evaluate how their personal life experiences and choices fit within the larger mosaic of U.S. society by confronting and critically analyzing their own values and assumptions about individuals and groups from different cultural contexts.
\item
  Understand how questions of diversity intersect with moral and political questions of justice and equality.
\end{enumerate}

\hypertarget{canvas}{%
\section{Canvas}\label{canvas}}

\textbf{Canvas} is a learning management system similar to Google Classroom and Blackboard.

\begin{rmdwarning}
This section will be updated before the first day of class.
\end{rmdwarning}

\hypertarget{readings}{%
\section{Readings}\label{readings}}

There are two books required for this course. Each book has been selected to correspond with one or more of the course objectives. The books are:

\begin{enumerate}
\def\labelenumi{\arabic{enumi}.}
\tightlist
\item
  Abraham, Laurie K. 2019. \emph{Mama Might Be Better Off Dead: The Failure of Health Care in Urban America}. Chicago, IL: The University of Chicago Press. ISBN-13: 978-0226623702; List Price: \$20.00; e-book versions available.

  \begin{itemize}
  \tightlist
  \item
    I do not require students to buy physical copies of \emph{Mama Might Be Better Off Dead.} You are free to select a means for accessing \emph{Mama} that meets your budget and learning style.
  \end{itemize}
\item
  Khan, Shamus, Patrick Sharkey, and Gwen Sharp, eds.~\emph{A Sociology Experiment}; e-book only.

  \begin{itemize}
  \tightlist
  \item
    This is a unique ``textbook'' where you purchase it by chapter instead of buying the entire book. Each chapter is written by one or more leading sociologists in the fields that the chapter covers. Each chapter costs \$1. If you have a concern about accessing this non-traditional resource, please let me know as soon as possible.
  \item
    I recommend purchasing all of the chapters at once rather than one at a time to reduce any associated credit card fees.
  \item
    Once you purchase a chapter, you will be able to download a \texttt{.pdf} copy of the text to keep.
  \item
    You will need to purchase the following Chapters:

    \begin{itemize}
    \tightlist
    \item
      All of Part 1- ``1 - A Sociology Experiment,'' ``2 - Research Methods,'' and ``3- Social Structure and the Individual''
    \item
      All of Part 2 - ``4 - Social Class, Inequality, and Poverty,'' ``5 - Culture,'' ``6 - Gender and Sexuality,'' and ``7 - Race and Ethnicity''
    \item
      From Part 3 - ``12 - Urban Sociology'' and ``15 - Health and Illness''
    \end{itemize}
  \end{itemize}
\end{enumerate}

\begin{rmdwarning}
Only \emph{Mama Might Be Better Off Dead} is available through the
bookstore! Use this link or the link below to access \emph{A Sociology
Experiment}! All chapters can be purchased through the text's website.
\end{rmdwarning}

All readings are listed on the \href{/syllabus/lecture-schedule.html}{\textbf{Reading List}} and should be completed before the course meeting on the week in which they are assigned (unless otherwise noted).

Many of the readings on the syllabus are peer reviewed journal articles that you'll be able to access through the library. Being ``peer reviewed'' means that they are written with an aim to contribute to scientific debates. Their primary audiences are typically health care providers, professors, and graduate students. They are therefore sometimes \emph{difficult} to read. Give yourself time - I don't expect each student to fully understanding the intricacies of each article (especially the statistics included in some), but I do expect you to walk away with a general sense of the argument and evidence presented.

\hypertarget{success-in-soc-1120}{%
\section{Success in SOC 1120}\label{success-in-soc-1120}}

Since this course will meet remotely for the entire semester, I have posted several articles on the Course Docs that include suggestions for navigating the course content, using Zoom, and using Canvas. Please review these before the semester begins. Following these will be crucial for having a successful semester!

Students often ask me how to do well in various aspects of the course, and so I also have \emph{suggestions} for a successful semester. \emph{These observations are provided with no warranty} - following them does not guarantee any particular outcome. You could do everything in here and still do poorly in the course, and conversely you could ignore much of what is in discussed in the links below and still do well. However, \emph{most} of the students who are successful in this course will follow \emph{most} of these ideas consistently.

\begin{enumerate}
\def\labelenumi{\arabic{enumi}.}
\tightlist
\item
  Doing the Little Things Right
\item
  Come to Office Hours!
\item
  Studying for Quizzes
\item
  Writing in the Social Sciences
\item
  Letters of Recommendation - Pay particular to the section titled ``If I Say No'' if you may want an Instructor Evaluation for medical school - I prioritize letter writing for students who I get to know outside of the classroom.
\end{enumerate}

These tips are an effort to illuminate what sociologists refer to as the ``\href{https://books.google.com/books?hl=en\&lr=\&id=5r-TAgAAQBAJ\&oi=fnd\&pg=PP1\&dq=hidden+curriculum\#v=onepage\&q=hidden\%20curriculum\&f=false}{hidden curriculum}'' of higher education - there are things you need to do to be successful, but they are often unstated or not clearly communicated. If there are other topics you have questions about, please let me know. These documents are a work in progress.

\begin{rmdnote}
Part of these documents involve tips for emailing faculty and link to an
article with some additional tips. If you send me an email prior to the
second lecture that breaks as many rules as possible (but clearly
identifies in some way who you are), I'll give you extra credit.
\end{rmdnote}

\hypertarget{course-policies}{%
\chapter{Course Policies}\label{course-policies}}

My priority is that class periods are productive learning experiences for all students. In order to foster this type of productive environment, I ask students to follow a few general policies and expectations:\footnote{These general expectations were adopted from language originally used by Dr.~Shelley Kimmelberg.}

\begin{enumerate}
\def\labelenumi{\arabic{enumi}.}
\tightlist
\item
  Work each week to contribute to a positive, supportive, welcoming, and compassionate class environment.
\item
  Arrive to class on time and stay for the entire class period.
\item
  Silence \emph{all} electronic devices before entering the classroom.
\item
  Do not engage in side conversations. This is disrespectful to the speaker (whether me or a classmate), and can affect the ability of others in the class to learn.
\item
  Be respectful of your fellow classmates. Do not interrupt when someone is speaking, monopolize the conversation, or belittle the ideas or opinions of others.
\item
  Complete the assigned readings for each class in advance, and come prepared with discussion points and questions.
\item
  Follow my best practices for using Zoom and Canvas.
\end{enumerate}

The following sections contain additional details about specific course policies related to attendance, participation, electronic device use, student support, academic honesty, and Title IX.

\hypertarget{covid-19}{%
\section{COVID-19}\label{covid-19}}

We are in the midst of something very few Americans alive today have experienced before to this degree - a pandemic caused by a highly contagious virus called ``SARS-CoV-2.'' This virus causes an illness called COVID-19. Not since the fall of 1919 have Saint Louis University students begun a semester quite like this one. I think acknowledging that we are all starting from varying places of exhaustion, stress, and anxiety is critical. To that end, I've written an open letter to all of you that I ask you to read before our first class.

Please, first and foremost, focus on what matters most and practice self care. Please follow SLU's guidance both for your own health and for maintaining community. If you are living on-campus, please follow all of SLU's policies for social distancing and mask wearing. These are critical for our collective safety. If you are off campus this semester, please follow them anyway!

Again, my biggest priority this semester is your health and well-being. Please reach out if you want to talk about strategies for managing our ``new normal,'' if you find yourself struggling, or just need someone to vent to. Alternatively, you can reach out to either the University Counseling Center at 314-977-TALK or Campus Ministry at 314-977-2425 or \href{mailto:campusministry@slu.edu}{\nolinkurl{campusministry@slu.edu}}.

\hypertarget{planning-for-disruptions}{%
\subsection{Planning for Disruptions}\label{planning-for-disruptions}}

While we are starting the semester with all of you in St.~Louis, and I am certainly rooting for a semester where you all are able to remain on-campus through Thanksgiving Break, we should recognize that there may be disruptions. For some of you, you may find yourselves quarantined because of an exposure to someone infected with COVID-19. You may become sick yourselves. Changes both on-campus and in the greater St.~Louis community may mean changes to how SLU operates and even whether or not you can continue to remain on-campus.

Given that these are not abstract concerns, everything laid out here represents a best case scenario for the semester. We may find ourselves needing to change some course policies, reading and assignment schedules, and even teaching modalities based on the challenges we are confronted with this semester. I ask for your patience and your flexibility if and when we do need to make these changes. For my part, I will do my best to stay in touch with you and communicate clearly how these changes will impact our course.

One way that I am proactively preparing for disruptions is to add two \textbf{``flex days''} to the syllabus. I do not have any content planned for \textbf{October 26} or \textbf{November 23}. If there is a widespread disruption, such as the need for many of you to move off campus, I will use these flex days as a buffer so that we can adjust our course schedule without changing the basic structure of the course. If we are approaching these dates without having had to use them, these will become days off from this course.

\textbf{You should therefore treat all of the course dates as provisional.} This is my plan as of August 20, and I may modify it further as we progress through the the semester. If I become sick or a member of my family becomes ill, modifications will likely be required. If another faculty member has to take over teaching my class, there may be changes to course content, teaching modality, and assignments. I will do my best to keep everyone updated in a timely fashion. Please check your email and Canvas regularly. I appreciate everyone's willingness to roll with the many punches we are all facing right now. Remember, we are in this together.

\hypertarget{face-masks}{%
\subsection{Face Masks}\label{face-masks}}

Throughout the COVID-19 pandemic, key safeguards like face masks have allowed SLU to safely maintain in-person learning. If public health conditions and local, state, and federal restrictions demand it, the University may require that all members of our campus community wear face masks indoors.

\textbf{Therefore, any time a University-level face mask requirement is in effect, face masks will be required in this class.} This expectation will apply to all students and instructors, unless a medical condition warrants an exemption from the face mask requirement (see below).

\textbf{When a University-wide face mask requirement is in effect,} the following will apply:

\begin{itemize}
\tightlist
\item
  Students who attempt to enter a classroom without wearing masks will be asked by the instructor to put on their masks prior to entry. Students who remove their masks during a class session will be asked by the instructor to resume wearing their masks.
\item
  Students and instructors may remove their masks briefly to take a sip of water but should replace masks immediately. The consumption of food will not be permitted.
\item
  Students who do not comply with the expectation that they wear a mask in accordance with the University-wide face mask requirement may be subject to disciplinary actions per the rules, regulations, and policies of Saint Louis University, including but not limited to those outlined in the Student Handbook. Non-compliance with this policy may result in disciplinary action, up to and including any of the following:

  \begin{itemize}
  \tightlist
  \item
    dismissal from the course(s)
  \item
    removal from campus housing (if applicable)
  \item
    dismissal from the University
  \end{itemize}
\item
  To immediately protect the health and well-being of all students, instructors, and staff, instructors reserve the right to cancel or terminate any class session at which any student fails to comply with a University-wide face mask requirement.
\end{itemize}

\textbf{When a University-wide face mask requirement is in effect,} students and instructors may choose to wear a face mask or not, as they prefer for their own individual comfort level.

\hypertarget{ada-accomodations-for-face-mask-requirements}{%
\subsubsection{ADA Accomodations for Face Mask Requirements}\label{ada-accomodations-for-face-mask-requirements}}

Saint Louis University is committed to maintaining an inclusive and accessible environment. Individuals who are unable to wear a face mask due to medical reasons should contact the Office of Disability Services (students) or Human Resources (instructors) to initiate the accommodation process identified in the University's \href{https://www.slu.edu/human-resources/pdfs/policies/americans-disabilities-act-policy.pdf}{ADA Policy}. Inquiries or concerns may also be directed to the \href{https://www.slu.edu/general-counsel/institutional-equity-diversity/index.php}{Office of Institutional Equity and Diversity}. Notification to instructors of SLU-approved ADA accommodations should be made in writing prior to the first class session in any term (or as soon thereafter as possible).

\hypertarget{attendance-policies}{%
\subsection{Attendance Policies}\label{attendance-policies}}

The health and well-being of SLU's students, staff, and faculty are critical concerns, as is the quality of our learning environments. Accordingly, the following University policy statements on in-person class attendance are designed to preserve and advance the collective health and well-being of our institutional constituencies and to create the conditions in which all students have the opportunity to learn and successfully complete their courses.

\begin{enumerate}
\def\labelenumi{\arabic{enumi}.}
\tightlist
\item
  Students who exhibit \href{https://www.cdc.gov/coronavirus/2019-ncov/symptoms-testing/symptoms.html}{any potential COVID-19 symptoms} (those that cannot be attributed to some other medical condition the students are known to have, such as allergies, asthma, etc.) shall absent themselves from any in-person class attendance or in-person participation in any class-related activity until they have been evaluated by a qualified medical official. Students should contact the \href{https://www.slu.edu/life-at-slu/student-health/index.php}{University Student Health Center} for immediate assistance.
\item
  Students (whether exhibiting any of potential COVID-19 symptoms or not, and regardless of how they feel) who are under either an isolation or quarantine directive issued by a qualified health official must absent themselves from all in-person course activities per the stipulations of the isolation or quarantine directive.
\item
  Students are responsible for notifying their instructor of an absence as far in advance as possible; when advance notification is not possible, students are responsible for notifying each instructor as soon after the absence as possible. Consistent with the \href{https://catalog.slu.edu/academic-policies/academic-policies-procedures/attendance/}{University Attendance Policy}, students also are responsible for all material covered in class and must work with the instructor to complete any required work. In situations where students must be absent for an extended period of time due to COVID-19 isolation or quarantine, they also must work with the instructor to determine the best way to maintain progress in the course as they are able based on their health situation.
\item
  Consistent with the \href{https://catalog.slu.edu/academic-policies/academic-policies-procedures/attendance/}{University Attendance Policy}, students may be asked to provide medical documentation when a medical condition impacts a student's ability to attend and/or participate in class for an extended period of time.
\item
  As a temporary amendment to the current \href{https://catalog.slu.edu/academic-policies/academic-policies-procedures/attendance/}{University Attendance Policy}, all absences due to illness or an isolation/quarantine directive issued by a qualified health official, or due to an adverse reaction to a COVID-19 vaccine, shall be considered ``Authorized'' absences.
\end{enumerate}

\hypertarget{seating-policy}{%
\subsection{Seating Policy}\label{seating-policy}}

In order to facilitate contact tracing, the University is requiring that we record where students sit each day of class. In order to reduce the burden of this on all of us, I will ask that you sit in the same space for each class section. For the first three course meetings, I will ask students to sign-in not just for attendance purposes, but for seating location purposes as well. After the third course meeting, I will ask you to not switch seats for the remainder of the semester. If you need to change seating for any reason during the remainder of the semester, please speak with me first. When we have our \emph{Mama} discussions, I will ask each group to fill out an individual seating chart to facilitate contact tracing. I recognize that this is an added burden for all of us, and appreciate your cooperation with it.

\hypertarget{zoom-policies}{%
\subsection{Zoom Policies}\label{zoom-policies}}

\begin{rmdwarning}
This section of the syllabus will only apply if we switch to having
class remotely.
\end{rmdwarning}

If we have to meet via Zoom, there will be several additional policies to note:

\begin{enumerate}
\def\labelenumi{\arabic{enumi}.}
\tightlist
\item
  Attending via Zoom is required. There is not an alternative means for completing this course. \emph{If you have a concern about technology, internet access, or other barriers to regularly attending class via Zoom, please let me know as soon as possible.}

  \begin{itemize}
  \tightlist
  \item
    If there is a need for some or all of you to change from learning on-campus to learning from home or another location, we will work together to identify strategies for you to successfully complete the course.
  \end{itemize}
\item
  Do not share Zoom details, including login information, links, and passwords, with anyone outside of this course.
\item
  Using your camera is \emph{strongly encouraged} during group discussions, but is not otherwise required.
\item
  Please keep your microphone muted unless you are actively speaking.
\item
  Class recordings will be made via Zoom and/or Panopto, and will be posted to Canvas. Recordings should not be shared outside of class. They will capture whatever is happening on my screen, which may include sharing your name and whatever is actively shown via your webcam. \emph{If this presents a privacy concern for you, please let me know as soon as possible.}
\end{enumerate}

The Course Docs contain some additional tips for using Zoom. Please review them closely.

\hypertarget{attendance-and-participation}{%
\section{Attendance and Participation}\label{attendance-and-participation}}

\hypertarget{general-attendance-policy}{%
\subsection{General Attendance Policy}\label{general-attendance-policy}}

Attendance and participation are important components of this course. Your expected to attend all class sessions and to arrive before the beginning of class. That said, it is important to recognize that our normal attendance policies are not well suited to a pandemic. If you cannot attend class or arrive on time because of a personal illness, a family issue, jury duty, an athletic match, or a religious observance, you must contact me \textbf{beforehand} to let me know if at all possible. I define family issues broadly - if your family or friends become sick or are being affected by COVID-19 in other ways, please know that I want you to keep your focus on what is most important.

I may ask for more information, such as a note from a health care provider, a travel letter from Athletics, or other documentation for absences. I will not be asking for health care provider documentation for acute illnesses or injuries, though, since if you're sick but not \emph{very} sick, the last thing most you will want to do is go to a doctor just to get a note. I am proceeding with a spirit of trust in all of you, and ask you reciprocate that with me. If you need to modify assignment due dates, please let me know prior to those deadlines.

Please see the University's attendance policy for additional details.

\hypertarget{attendace-collection}{%
\subsection{Attendace Collection}\label{attendace-collection}}

In order to help identify students who might need extra support, I do keep track of who attends class. Attendance check-ins will be collected through a simple web-form. Students will need a QR code reader application installed on their smartphone to check-in if their phone does not support it automatically. The Course Docs contain some additional details and links for learning more about how this works.

This QR code will be available as print-outs in the back and front of the classroom. Please scan them as you come into class. Attendance submissions must be received by the end of the class period.

These web forms are \textbf{time stamped}, so if you sign the form 3 minutes after the beginning of class or later, you will be marked as `late' in the attendance database. This is done automatically by my gradebook, so please see me if you have a concern about how this works or, more generally, if you have a concern about regularly being able to arrive to class on time.

If you do not own a smartphone, please let me know as soon as possible. You should note that attendance check-ins are covered by the course's Academic Honesty policy. Sharing the check-in form with another student or signing in on their behalf are both violations of this policy.

\begin{rmdwarning}
If we have to switch to having class remotely, the QR code will be
available on-screen. This means that you need to arrive to class at
least a minute or two before we begin. If you arrive after the QR code
has been taken down, I'll put it up again at the end of the ``front
matter'' and ``back matter'' sections of the lecture slides. Attendance
submissions must be received by the end of the class period.
\end{rmdwarning}

\hypertarget{missed-classes}{%
\subsection{Missed Classes}\label{missed-classes}}

My priority with attendance is to identify students who may be struggling or in need of additional support. However, because attending class is crucial, I do factor attendance into your overall participation grade. In order to give you some flexibility, I do not apply any penalties to your first two unexcused absences. Any unexcused absences beyond those two will result in no credit (for an absences). Regular late arrivals may result in partial credit being earned for that day's participation grade.

Specific elements of the course, such as participation in group discussions during lectures as well as the QHQ discussions, will also be factored into your participation grade. If there is a need for you to miss a significant portion of your coursework, such as because of an illness, please reach out to me and we will make a plan for alternative ways to make-up these activities based on the circumstances.

It is your responsibility to make-up missed classes, including viewing the lecture recording and, if needed, obtaining notes from a classmate All lecture slides will be posted on Canvas before class begins along with relevant notes for that lecture. Please note that lectures and discussions cannot be recorded by any means (e.g.~audio or video recordings, or photographs) without my permission.

\hypertarget{communication}{%
\section{Communication}\label{communication}}

Email is my preferred method of communication. I dedicate time to email responses each workday, meaning that my response time is typically within 24 hours during the workweek. If you have not received a response from me after 48 hours (or by end of business on Monday if you emailed me over the weekend), please feel free to follow-up with me.

Please use your SLU email account when emailing me. All messages regarding course updates, assignments, and changes to the class schedule including cancellations will be sent to your SLU email account. It is therefore imperative that you check your SLU email account regularly.

Please also ensure that all concerns or questions about your standing in the course are directed to me immediately. Inquires from parents, SLU staff members, and others will not be honored.

\hypertarget{electronic-devices}{%
\section{Electronic Devices}\label{electronic-devices}}

During class periods, students are asked to refrain from using electronic devices (including cell phones) for activities not directly related to the course. For this class, I expect students to limit their use of electronic devices to accessing Zoom, course readings, notes, and other course materials.

There is evidence that using electronic devices during lectures results in decreased retention of course content (\href{https://link.springer.com/article/10.1007/BF02940852}{Hembrooke and Gay 2003}) and lower overall course performance (\href{https://www.sciencedirect.com/science/article/pii/S0360131506001436}{Fried 2008}). Students who are not using a laptop but are in direct view of another student's laptop also have decreased performance in courses (\href{https://www.sciencedirect.com/science/article/pii/S0360131512002254}{Sana et al.~2013}). Conversely, students who take notes the ``old fashioned way'' have better performance on tests compared to students who take notes on laptops (\href{http://journals.sagepub.com/doi/abs/10.1177/0956797614524581}{Mueller and Oppenheimer 2014}).

I therefore ask students to be conscious of how they are using their devices, the ways such use impacts their own learning, and the effect that it may have on others around them. I reserve the right to alter this policy if electronic device use becomes problematic during the semester.

\hypertarget{student-support}{%
\section{Student Support}\label{student-support}}

\hypertarget{basic-needs}{%
\subsection{Basic Needs}\label{basic-needs}}

If you have difficulty affording groceries or accessing sufficient food to eat every day, or lack a safe and stable place to live, you are urged to contact the \href{https://www.slu.edu/student-development/dean-of-students/index.php}{Dean of Students} for support. Likewise if you have concerns about your mental or physical health needs, or lack access to health care services you require, you should contact either the \href{https://www.slu.edu/student-development/dean-of-students/index.php}{Dean of Students}, \href{https://www.slu.edu/life-at-slu/student-health/index.php}{Student Health Services}, or the \href{https://www.slu.edu/life-at-slu/university-counseling/index.php}{University Counseling Center}.\footnote{This language is adopted from text written by \href{https://medium.com/@saragoldrickrab/basic-needs-security-and-the-syllabus-d24cc7afe8c9}{Dr.~Sarah Goldrick-Rab}.}

If you feel comfortable doing so, please discuss any concerns you might have with me. Doing so is particularly important if believe your performance in this course might be affected. I will do my best to work with you to come up with a plan for successfully completing the course and, if need be, work with you to identify on-campus resources. I will treat all discussions with discretion, though please be aware that certain situations, including disclosures of \href{/syllabus/harassment-and-title-ix.html}{sexual misconduct} or self harm, must be reported by faculty to the appropriate University office.

\hypertarget{academic-accommodations}{%
\subsection{Academic Accommodations}\label{academic-accommodations}}

If you meet the eligibility requirements for academic accommodations through the \href{https://www.slu.edu/life-at-slu/student-success-center/disability-services/index.php}{Office of Disability Services} (located within the Student Success Center) \emph{and you wish to use them for this class}, you should arrange to discuss your needs with me after the first class. All discussions of this nature are treated confidentially, and I will make every effort to work with you to come up with a plan for successfully completing the course requirements.

Please note that I will not provide accommodations to students who are not working with Disability Services, and that I cannot retroactively alter assignments or grades if they have already been completed. This follows the University policies on disability accommodations:

\begin{quote}
Students with a documented disability who wish to request academic accommodations must formally register their disability with the University. Once successfully registered, students also must notify their course instructor that they wish to use their approved accommodations in the course.
\end{quote}

\begin{quote}
Please contact Disability Services to schedule an appointment to discuss accommodation requests and eligibility requirements. Most students on the St.~Louis campus will contact Disability Services, located in the Student Success Center and available by email at \href{mailto:Disability_services@slu.edu}{\nolinkurl{Disability\_services@slu.edu}} or by phone at 314-977-3484. Once approved, information about a student's eligibility for academic accommodations will be shared with course instructors by email from Disability Services and within the instructor's official course roster. Students who do not have a documented disability but who think they may have one also are encouraged to contact to Disability Services. Confidentiality will be observed in all inquiries.
\end{quote}

\hypertarget{writing-services}{%
\subsection{Writing Services}\label{writing-services}}

I also encourage you to take advantage of the \href{https://www.slu.edu/life-at-slu/student-success-center/academic-support/university-writing-services/index.php}{University Writing Services (UWS) program}. Getting feedback benefits writers at all skill levels and the quality of your writing will be reflected in assignment grades. The UWS has trained writing consultants who can help you improve the quality of your written work. UWS's consultants are available to address everything from brainstorming and developing ideas to crafting strong sentences and documenting sources.

\hypertarget{student-success-coaching}{%
\subsection{Student Success Coaching}\label{student-success-coaching}}

Academic coaches are staff members who can assist with study skills, time management, test and note taking, goal setting, and motivations. They can also help deal with navigating homesickness, making connections on campus, and being successful in online/remote coursework. Coaches will work with you on a weekly basis to develop the skills that are most important to you. For more information, please contact Emily Tuttle.

\hypertarget{academic-honesty}{%
\section{Academic Honesty}\label{academic-honesty}}

All students should familiarize themselves with \href{https://www.slu.edu/provost/policies/academic-and-course/policy_academic-integrity_6-26-2015.pdf}{Saint Louis University's policies} the the \href{https://www.slu.edu/arts-and-sciences/student-resources/academic-honesty.php}{College of Arts and Sciences policies} concerning cheating, plagiarism, and other academically dishonest practices:

\begin{quote}
Academic integrity is honest, truthful and responsible conduct in all academic endeavors. The mission of Saint Louis University is ``the pursuit of truth for the greater glory of God and for the service of humanity.'' Accordingly, all acts of falsehood demean and compromise the corporate endeavors of teaching, research, health care, and community service through which SLU fulfills its mission. The University strives to prepare students for lives of personal and professional integrity, and therefore regards all breaches of academic integrity as matters of serious concern.
\end{quote}

Any work that is taken from another student, copied from printed material, or copied the internet without proper citation is expressly prohibited, and will be addressed by the instructor. Collaborating on quizzes, such as taking them in groups (whether in-person or virtually), is also prohibited. Any student who is found to have been academically dishonest in their work risks failing both the assignment and this course.

All relevant assignments should include in-text citations and references formatted using the \href{https://owl.english.purdue.edu/owl/resource/583/1/}{American Sociological Association (ASA)} style guidelines.

\hypertarget{harassment-and-title-ix}{%
\section{Harassment and Title IX}\label{harassment-and-title-ix}}

While I have every expectation that each member of the Saint Louis University community is capable and willing to create a positive coursework experience, I fully recognize that there may be instances where students fall short of that expectation. Students should generally be aware that:

\begin{quote}
Saint Louis University prohibits harassment because of sex, race, color, religion, national origin, ancestry, disability, age, sexual orientation, marital status, military status, veteran status, gender expression/identity, genetic information, pregnancy, or any other characteristics protected by law.
\end{quote}

All students should also familiarize themselves with \href{http://www.slu.edu/general-counsel/institutional-equity-diversity/}{Saint Louis University's polices} on bias, discrimination, harassment, and sexual misconduct. In particular, they should be aware of policies on \href{http://www.slu.edu/general-counsel/institutional-equity-diversity/harassment.php}{harassment} and \href{https://www.slu.edu/about/safety/sexual-assault-resources.php}{sexual misconduct}:

\begin{quote}
Saint Louis University and its faculty are committed to supporting our students and seeking an environment that is free of bias, discrimination, and harassment. If you have encountered any form of sexual harassment, including sexual assault, stalking, domestic or dating violence, we encourage you to report this to the University. If you speak with a faculty member about an incident that involves a Title IX matter, that faculty member must notify SLU's Title IX Coordinator and share the basic facts of your experience. This is true even if you ask the faculty member not to disclose the incident. The Title IX Coordinator will then be available to assist you in understanding all of your options and in connecting you with all possible resources on and off campus.
\end{quote}

\begin{quote}
Anna Kratky is the Title IX Coordinator at Saint Louis University (DuBourg Hall, Room 36; \href{mailto:anna.kratky@slu.edu}{\nolinkurl{anna.kratky@slu.edu}}; 314-977-3886). If you wish to speak with a confidential source, you may contact the counselors at the University Counseling Center at 314-977-TALK or make an anonymous report through SLU's Integrity Hotline by calling 1-877-525-5669 or online at \url{https://www.lighthouse-services.com/_StandardCustomURL/LHILandingPage.asp}. To view SLU's policies, and for resources, please visit the following web addresses: \url{https://www.slu.edu/here4you} and \url{https://www.slu.edu/general-counsel}.
\end{quote}

Instances of abusive, harassing, or otherwise unacceptable behavior should be reported either directly to the instructor or to the University Administration. Consistent with the above policies, I will forward all reports of inappropriate conduct to the Title IX Coordinator's office or to the Office of Diversity and Affirmative Action. Please be aware that University policies may require me to forward information about the identity of any students connected to the disclosure.

Please also be aware that communications over various online services, including (but not limited to) Canvas and Zoom, are covered by this policy.

\hypertarget{assignments-and-grading}{%
\chapter{Assignments and Grading}\label{assignments-and-grading}}

This section provides general details on the different types of assignments for this course. It also contains policies for submitting work, receiving feedback, and late work. A summary schedule with all due dates is available as part of the \href{course-schedule.html}{Course Schedule}.

\begin{rmdwarning}
Students enrolled in the honors section should see
\href{/syllabus/honors-overview.html}{Section 6} for additional
assignment descriptions as well as their weighting and final point
totals.
\end{rmdwarning}

\hypertarget{assignments}{%
\section{Assignments}\label{assignments}}

Your grade for this course will consist of a number of different assignments on which points may be earned. Each category of assignment is described below.

\begin{rmdwarning}
As a reminder, all due dates are provisional due to the uncertainty
around COVID-19 and how it will affect our semester. I will continually
update you during class and via Canvas about the status of each
assignment's due dates as the semester progresses.
\end{rmdwarning}

\hypertarget{attendance-and-participation-1} of
your final grade
\end{rmdtip}

As discussed above, both attendance and participation are important aspects of this class. The class participation grade will be based on (a) attendance, (b) level of engagement during class (including being present for and participating in course discussions), and (c) class ``entry'' and ``exit tickets.''

Each of these elements is assigned a point value and assessed using a scale that awards full, partial, or no credit (see \href{grading.html}{Grading}). Not attending class or completing an ``entry'' or ``exit'' ticket will result in no credit being earned for that element on a given day. Disengagement during class may result in partial or no credit being earned. Late arrivals will result in only partial credit earned for that element on a given day.

Your participation grade will be split, with 20 points for Part 1, 30 points each for Parts 2 and 3, and 15 points for Part 4. Since the number of points awarded for participation are variable, the total number of points earned for each half will be \textbf{weighted} so that it is converted to a final score that matches the points available for that part of the course. I provide the final number of points earned for each part of the course. If you would like a more detailed breakdown of your participation grade and/or attendance record, please reach out and I will happily provide one.

\hypertarget{theory-isnt-dead-posts} of your final grade
\end{rmdtip}

Over the course of the semester, there will be four short exercises that ask you to tie social theory, social science research, and current events together. For each of these assignments, I will post a news article. Based on your read of the article, you should pick one of the social theories we have discussed in class or in readings, briefly describe why you selected that theory (in 2-3 sentences), and then craft a theoretically motivated research question and hypothesis (a sentence each). Finally, you will briefly describe how you would propose to study this hypothesis (in 2-3 sentences). Your assignments will be posted to Canvas before 5pm on the due date.

Additional details and a sample assignment will be made available via Canvas. Each Theory Post is worth 20 points. Both elements will be assessed using a scale that awards full, partial, or no credit (see \href{grading.html}{Grading}).

Due dates for the Theory Post are as follows:

\begin{enumerate}
\def\labelenumi{\arabic{enumi}.}
\tightlist
\item
  Theory Isn't Dead 1 - \textbf{Thursday, September 9}
\item
  Theory Isn't Dead 2 - \textbf{Tuesday, October 5}
\end{enumerate}

\hypertarget{sociological-experiences} of your final
grade
\end{rmdtip}

Over the course of the semester, there will be four short exercises that ask you to link data to your own personal experiences. Data for these exercises will be drawn primarily from Social Explorer, a tool that makes it easy to find demographic data about various places in the United States. These reflections should be approximately 1-2 paragraphs in length.

Additional details and a sample assignment will be made available via Canvas. Each Sociological Experience is worth 20 points, and will be assessed using a scale that awards full, partial, or no credit (see \href{grading.html}{Grading}). Assignments are due by the beginning of class on the due date.

Due dates for the Sociological Experiences are as follows:

\begin{enumerate}
\def\labelenumi{\arabic{enumi}.}
\tightlist
\item
  Sociological Experience 1 - High School - \textbf{Tuesday, September 21}
\item
  Sociological Experience 2 - Socioeconomic Status - \textbf{Tuesday, October 12}
\item
  Sociological Experience 3 - Segregation - \textbf{Thursday, November 11}
\item
  Sociological Experience 4 - St.~Louis - \textbf{Thursday, December 2}
\end{enumerate}

\hypertarget{mama-papers} of your final grade
\end{rmdtip}

Each student will write a reflection paper on three chapters (one chapter per discussion period) of \emph{Mama Might Be Better Off Dead} (Abraham 1993). These reflection papers will integrate previous lecture material and readings to understand the cycle of events described in the book. Additional details and a grading rubric will be available on Canvas. Papers must be completed and submitted by the beginning of class on the date due. Each paper is worth 50 points.

Due dates for the QHQs are as follows:

\begin{enumerate}
\def\labelenumi{\arabic{enumi}.}
\tightlist
\item
  \emph{Mama} Paper 1 - \textbf{Thursday, October 14}
\item
  \emph{Mama} Paper 2 - \textbf{Thursday, November 18}
\item
  \emph{Mama} Paper 3 - \textbf{Tuesday, December 7}
\end{enumerate}

\hypertarget{quizzes} of your final grade
\end{rmdtip}

Three non-cumulative multiple choice quizzes will be given throughout the semester. Each quiz will cover the breadth of the material in the course, including readings, lectures, and videos. Each quiz will consist of 30 multiple choice questions and will be worth 50 points. They will be administered via Canvas. Quizzes must be taken between 7am CST and 10pm CST on the quiz date. They can be started at any point within that range, but once you begin the quiz, you will have only 45 minutes to complete it. Since they occur remotely, quizzes are open book and note.

Quiz dates are as follows:

\begin{enumerate}
\def\labelenumi{\arabic{enumi}.}
\tightlist
\item
  Quiz 1 - \textbf{Monday, September 27}
\item
  Quiz 2 - \textbf{Monday, October 25}
\item
  Quiz 3 - \textbf{Tuesday, December 14}

  \begin{itemize}
  \tightlist
  \item
    This is overlaps with our scheduled final exam time, so you can take the quiz during that period if you wish.
  \end{itemize}
\end{enumerate}

\hypertarget{final-project} of your final
grade
\end{rmdtip}

This project involves the creation of a single page handout that illustrates the origins of a health disparity present in American society. On the handout, you should provide some background information, data about the disparity, and connect the disparity to fundamental cause theory and/or the social determinants of health perspective. You have creative license to design the handout as you see fit, and should feel free to add graphs, maps, or photographs that help you illustrate the issues you are describing. The second page of the handout should contain a list of works cited. More details and a rubric will be provided on Canvas. The final project will be due by \textbf{Tuesday, December 14}.

\hypertarget{submission-and-late-work}{%
\section{Submission and Late Work}\label{submission-and-late-work}}

\hypertarget{canvas-submissions}{%
\subsection{Canvas Submissions}\label{canvas-submissions}}

All assignments must be submitted via Canvas \emph{as a \texttt{.pdf} file} unless otherwise noted in the assignment instructions. Feedback will be returned to students via comments embedded in each document.

\hypertarget{late-work}{%
\subsection{Late Work}\label{late-work}}

Once the due date has passed, any assignments shared will be treated as late. Be advised that Canvas time-stamps submissions, so that even being a few seconds over the due date and time will result in your assignment being marked late. Like arriving late to class, this happens automatically, so please let me know as soon as possible \textbf{before} a due date if you have a concern about a potentially late submission.

Assignments shared within 24-hours of the due date will have 15\% deducted from the grade. I will deduct 15\% per day for the next two 24-hour periods that assignments are late; after 72-hours, I will not accept late work. If you cannot submit work on time because of a personal illness, a family issue, jury duty, an athletic match, or a religious observance, you must contact me \emph{beforehand if at all possible} to discuss alternate submission of work. I may ask for more information, such as a note from a physician, a travel letter from Athletics, or other documentation for alternative deadlines.

\hypertarget{missed-quizzes}{%
\subsection{Missed Quizzes}\label{missed-quizzes}}

If you cannot attend an exam due to a personal illness, a family issue, jury duty, an athletic match, or a religious observance, you must contact me \textbf{beforehand} to discuss alternate quiz scheduling. I may ask for more information, such as a note from a physician, a travel letter from Athletics, or other documentation for rescheduled quizzes.

\hypertarget{extra-credit}{%
\section{Extra Credit}\label{extra-credit}}

From time to time I may offer extra credit to be applied to your final grade. I will only offer extra credit if it is open to the entire class (typically for something like attending a lecture or event on-campus). If I offer extra credit, I will generally require you to submit a short written summary of the activity within a week of the event to obtain the credit. Papers should be submitted via Canvas and will typically consist of a paragraph describing the event and a paragraph connecting the event to the class material in some way. When offered, extra credit opportunities cannot be made-up or substituted if you are unable to attend the event.

\hypertarget{grading}{%
\section{Grading}\label{grading}}

\begin{rmdwarning}
Students enrolled in the honors section should see
\href{/syllabus/honors-overview.html}{Section 7} for additional
assignment descriptions as well as their weighting and final point
totals.
\end{rmdwarning}

All grades that use a ``check'' system (the Theory Isn't Dead and Sociological Experience assignments as well as ``entry'' and ``exit'' tickets) will be calculated using the following approach. A ``check-plus-plus'' represents exceptional work and will get full credit. A ``check-plus'' represents great work and will get 92\% of the points available for the assignment. A ``check'' represents satisfactory work and will get 85\% of the points available for that assignment. A ``check-minus'' represents work that needs substantial improvement and will get 75\% of the points available for that assignment. For other assignments, rubrics will be provided prior to the due date that break down how grading works.

I use a point system for calculating grades. The following table gives the weighting and final point totals for all assignments for this course:

\begin{table}

\caption{\label{tab:unnamed-chunk-11}SOC 1120 Points Breakdown}
\centering
\begin{tabular}[t]{llllll}
\toprule
Assignment & Period & Points & Quantity & Total & Percent\\
\midrule
Participation & Part 1 & 20 pts & x1 & 20 pts & 16.1\%\\
 & Part 2 & 30 pts & x1 & 30 pts & \\
 & Part 3 & 30 pts & x1 & 30 pts & \\
 & Part 4 & 15 pts & x1 & 15 pts & \\
Theory Isn't Dead &  & 20 pts & x2 & 40 pts & 6.8\%\\
\addlinespace
Sociological Experiences &  & 20 pts & x4 & 80 pts & 13.6\%\\
QHQs &  & 75 pts & x2 & 150 pts & 25.4\%\\
Quizzes &  & 50 pts & x3 & 150 pts & 25.4\%\\
Final Project &  & 75 pts & x1 & 75 pts & 12.7\%\\
\bottomrule
\end{tabular}
\end{table}

All feedback will include grades that represent number of points earned. If you want to know your percentage on a particular assignment, divide the number of points earned by the number of points possible and then multiply it by 100.

Some of the provided rubrics on Canvas result in final points for assignments that include decimals. In the event of non-standard decimals (those other than .25, .5, or .75), I will round your grade up to the next standard decimal value (e.g.~.25, .5, or .75).

\hypertarget{conflicting-or-incorrect-grades}{%
\subsection{Conflicting or Incorrect Grades}\label{conflicting-or-incorrect-grades}}

If you notice a discrepancy between the grade you received in the feedback and what appears on Canvas, please let me know as soon as possible. I will default to taking the higher of the two grades as the official grade.

\hypertarget{letter-grades}{%
\subsection{Letter Grades}\label{letter-grades}}

Letter grades will be calculated by taking the sum of all points earned and dividing it by the total number of points possible. This will be multiplied by 100 and then converted to a letter grade using the following table:

\begin{table}
\caption{\label{tab:unnamed-chunk-12}Course Grading Scale}

\centering
\begin{tabular}[t]{lll}
\toprule
GPA & Letter & Percent\\
\midrule
4.0 & A & 93.0\% - 100\%\\
3.7 & A- & 90.0\% - 92.9\%\\
3.3 & B+ & 87.0\% - 89.9\%\\
3.0 & B & 83.0\% - 86.9\%\\
2.7 & B- & 80.0\% - 82.9\%\\
\bottomrule
\end{tabular}
\centering
\begin{tabular}[t]{lll}
\toprule
GPA & Letter & Percent\\
\midrule
2.3 & C+ & 77.0\% - 79.9\%\\
2.0 & C & 73.0\% - 76.9\%\\
1.7 & C- & 70.0\% - 72.9\%\\
1.0 & D & 63.0\% - 69.9\%\\
0.0 & F & < 63.0\%\\
\bottomrule
\end{tabular}
\end{table}

Updates to grades will be provided at midterms, and you can follow your progress via the \texttt{My\ Grades} area on Canvas. I round-up final grades that are within a half percentage point of the next highest letter grade. Requests for final grade changes outside of this range will not be honored.

\begin{rmdwarning}
No chances will be given for revisions of poor grades. Incomplete grades
will be given upon request only if you have a ``C'' average and have
completed at least two-thirds of the possible points (412 points). You
should note that incomplete grades must be rectified by the specified
deadline or they convert to an ``F''. This policy reflects the
\href{https://catalog.slu.edu/academic-policies/academic-policies-procedures/incomplete-course/}{University's
policy on incomplete coursework}.
\end{rmdwarning}

\hypertarget{part-reading-list}{%
\part{Reading List}\label{part-reading-list}}

\hypertarget{course-schedule}{%
\chapter{Course Schedule}\label{course-schedule}}

The following is a high-level schedule that details the general topic covered by each module. Modules are collections of two or more lectures and class discussions under the broad headings provided below.

\begin{table}

\caption{\label{tab:unnamed-chunk-2}SOC 1120 Course Overview}
\centering
\begin{tabular}[t]{llll}
\toprule
Part & Module & Planned Start Date & Title\\
\midrule
**1** &  &  & **Thinking Like Sociologists**\\
1 & 1 & Thursday, August 26 & Engaging the Social World\\
1 & 2 & Thursday, September 9 & Structuring the Social World\\
**2** &  &  & **The Building Blocks of Society**\\
2 & 3 & Tuesday, September 28 & Culture\\
\addlinespace
2 & 4 & Tuesday, October 5 & Socioeconomic Status\\
2 & 5 & Tuesday, October 19 & Gender and Sexuality\\
**3** &  &  & **The Broken Heart of America**\\
3 & 6 & Tuesday, November 2 & Race and Racism\\
3 & 7 & Tuesday, November 16 & Intersectionality\\
\addlinespace
3 & 8 & Tuesday, November 30 & Urban Sociology\\
**4** &  &  & **Course Conclusion**\\
4 & 9 & Tuesday, December 7 & Course Conclusion\\
\bottomrule
\end{tabular}
\end{table}

\hypertarget{scheduling-notes}{%
\subsection{Scheduling Notes}\label{scheduling-notes}}

The course schedule may change as it depends on the progress of the class and the \href{covid-19.html}{challenges we are confronted by this semester}. The web version of this document will be updated to reflect any alterations, but the \texttt{.pdf} version will remain unaltered.

This semester, we will not have class on \textbf{October 28} and \textbf{November 25} because they fall on University breaks. Additionally, no class activities are scheduled for \textbf{October 26} or \textbf{November 23}. These are ``flex days,'' which I have left without a scheduled plan to accommodate changes due to COVID-19. If we are approaching these dates without having had to use them, these will become days off from this course.

\hypertarget{meeting-schedule}{%
\chapter{Meeting Schedule}\label{meeting-schedule}}

Select a module from the menu to see details about topics, readings, and assignments. Additional notes and links to course materials are available through Canvas, which has dedicated sections for each module and meeting.

\newpage

\hypertarget{module-1---engaging-the-social-world}{%
\section{Module 1 - Engaging the Social World}\label{module-1---engaging-the-social-world}}

\hypertarget{meeting-1-1---thursday-august-26---course-introduction}{%
\subsection*{Meeting 1-1 - Thursday, August 26 - Course Introduction}\label{meeting-1-1---thursday-august-26---course-introduction}}
\addcontentsline{toc}{subsection}{Meeting 1-1 - Thursday, August 26 - Course Introduction}

\begin{itemize}
\tightlist
\item
  \textbf{Before Class:}

  \begin{itemize}
  \tightlist
  \item
    Read \emph{A Sociology Experiment}, ``Chapter 1 - A Sociology Experiment,'' pp.~1-9 (Link)
  \item
    Complete the Course Onboarding tasks (Canvas)
  \end{itemize}
\item
  \textbf{After Class:}

  \begin{itemize}
  \tightlist
  \item
    Complete the Student Information Sheet for Tuesday, August 31 (Canvas)
  \end{itemize}
\end{itemize}

\begin{center}\rule{0.5\linewidth}{0.5pt}\end{center}

\hypertarget{meeting-1-2---tuesday-august-31---social-theory}{%
\subsection*{Meeting 1-2 - Tuesday, August 31 - Social Theory}\label{meeting-1-2---tuesday-august-31---social-theory}}
\addcontentsline{toc}{subsection}{Meeting 1-2 - Tuesday, August 31 - Social Theory}

\begin{itemize}
\tightlist
\item
  \textbf{Before Class:}

  \begin{itemize}
  \tightlist
  \item
    Read \emph{A Sociology Experiment}, ``Chapter 1 - A Sociology Experiment,'' pp.~9-24 (Link)
  \item
    Complete the Student Information Sheet (via Canvas)
  \end{itemize}
\end{itemize}

\begin{center}\rule{0.5\linewidth}{0.5pt}\end{center}

\hypertarget{meeting-1-3---thursday-september-2---theorizing-health-disparities}{%
\subsection*{Meeting 1-3 - Thursday, September 2 - Theorizing Health Disparities}\label{meeting-1-3---thursday-september-2---theorizing-health-disparities}}
\addcontentsline{toc}{subsection}{Meeting 1-3 - Thursday, September 2 - Theorizing Health Disparities}

\begin{itemize}
\tightlist
\item
  \textbf{Before Class:}

  \begin{itemize}
  \tightlist
  \item
    Read \emph{A Sociology Experiment}, ``Chapter 15 - Health and Illness,'' pp.~1-10 and pp.~23-35 (Link)
  \item
    Read Phelan et al.~(2010)

    \begin{itemize}
    \tightlist
    \item
      Phelan, Jo C., Bruce Link, and Parisa Tehranifar. 2010. ``Social Conditions as Fundamental Causes of Health Inequalities: Theory, Evidence, and Policy Implications.'' \emph{Journal of Health and Social Behavior} 51(S):S28-S40. (Link)
    \end{itemize}
  \item
    Watch \emph{Unnatural Causes}, Part 1 - ``In Sickness and in Wealth'' (Pius Library)
  \end{itemize}
\end{itemize}

\begin{center}\rule{0.5\linewidth}{0.5pt}\end{center}

\hypertarget{meeting-1-4---tuesday-september-7---studying-society}{%
\subsection*{Meeting 1-4 - Tuesday, September 7 - Studying Society}\label{meeting-1-4---tuesday-september-7---studying-society}}
\addcontentsline{toc}{subsection}{Meeting 1-4 - Tuesday, September 7 - Studying Society}

\begin{itemize}
\tightlist
\item
  \textbf{Before Class:}

  \begin{itemize}
  \tightlist
  \item
    Read \emph{A Sociology Experiment}, ``Chapter 2 - Research Methods'' (Link)
  \end{itemize}
\item
  \textbf{After Class:}

  \begin{itemize}
  \tightlist
  \item
    Complete Theory Isn't Dead 1 for Thursday, September 9 (Canvas)
  \end{itemize}
\end{itemize}

\newpage

\hypertarget{module-2---structuring-the-social-world}{%
\section{Module 2 - Structuring the Social World}\label{module-2---structuring-the-social-world}}

\hypertarget{meeting-2-1---thursday-september-9---nature-or-nurture}{%
\subsection*{Meeting 2-1 - Thursday, September 9 - Nature or Nurture?}\label{meeting-2-1---thursday-september-9---nature-or-nurture}}
\addcontentsline{toc}{subsection}{Meeting 2-1 - Thursday, September 9 - Nature or Nurture?}

\begin{itemize}
\tightlist
\item
  \textbf{Before Class:}

  \begin{itemize}
  \tightlist
  \item
    Read \emph{A Sociology Experiment}, ``Chapter 3 - Social Structure and the Individual,'' pp.~1-4 (Link)
  \item
    Read Bearman (2008)

    \begin{itemize}
    \tightlist
    \item
      Bearman, Peter. 2008. ``Introduction: Exploring Genetics and Social Structure.'' \emph{American Journal of Sociology} 114(S1):v-x. (Link)
    \end{itemize}
  \end{itemize}
\end{itemize}

\begin{center}\rule{0.5\linewidth}{0.5pt}\end{center}

\hypertarget{meeting-2-2---tuesday-september-14---social-structure}{%
\subsection*{Meeting 2-2 - Tuesday, September 14 - Social Structure}\label{meeting-2-2---tuesday-september-14---social-structure}}
\addcontentsline{toc}{subsection}{Meeting 2-2 - Tuesday, September 14 - Social Structure}

\begin{itemize}
\tightlist
\item
  \textbf{Before Class:}

  \begin{itemize}
  \tightlist
  \item
    Read \emph{A Sociology Experiment}, ``Chapter 3 - Social Structure and the Individual,'' pp.~4-14 (Link)
  \end{itemize}
\end{itemize}

\begin{center}\rule{0.5\linewidth}{0.5pt}\end{center}

\hypertarget{meeting-2-3---thursday-september-16---socialization}{%
\subsection*{Meeting 2-3 - Thursday, September 16 - Socialization}\label{meeting-2-3---thursday-september-16---socialization}}
\addcontentsline{toc}{subsection}{Meeting 2-3 - Thursday, September 16 - Socialization}

\begin{itemize}
\tightlist
\item
  \textbf{Before Class:}

  \begin{itemize}
  \tightlist
  \item
    Read \emph{A Sociology Experiment}, ``Chapter 3 - Social Structure and the Individual,'' pp.~15-26 (Link)
  \item
    Read Braveman and Barclay (2009)

    \begin{itemize}
    \tightlist
    \item
      Braveman, Paula and Colleen Barclay. 2009. ``Health Disparities Beginning in Childhood: A Life-Course Perspective.'' \emph{Pediatrics} 123(S3):S163-S175. (Link)
    \end{itemize}
  \end{itemize}
\item
  \textbf{After Class:}

  \begin{itemize}
  \tightlist
  \item
    Complete Sociological Experience 1 - High School for Tuesday, September 21 (Canvas)
  \end{itemize}
\end{itemize}

\begin{center}\rule{0.5\linewidth}{0.5pt}\end{center}

\hypertarget{meeting-2-4---tuesday-september-21---the-institution-of-medicine}{%
\subsection*{Meeting 2-4 - Tuesday, September 21 - The Institution of Medicine}\label{meeting-2-4---tuesday-september-21---the-institution-of-medicine}}
\addcontentsline{toc}{subsection}{Meeting 2-4 - Tuesday, September 21 - The Institution of Medicine}

\begin{itemize}
\tightlist
\item
  \textbf{Before Class:}

  \begin{itemize}
  \tightlist
  \item
    TBD
  \end{itemize}
\end{itemize}

\begin{center}\rule{0.5\linewidth}{0.5pt}\end{center}

\hypertarget{meeting-2-5---thursday-september-23---medicalization}{%
\subsection*{Meeting 2-5 - Thursday, September 23 - Medicalization}\label{meeting-2-5---thursday-september-23---medicalization}}
\addcontentsline{toc}{subsection}{Meeting 2-5 - Thursday, September 23 - Medicalization}

\begin{itemize}
\tightlist
\item
  \textbf{Before Class:}

  \begin{itemize}
  \tightlist
  \item
    Read \emph{A Sociology Experiment}, ``Chapter 15 - Health and Illness,'' pp.~10-16 (Link)
  \item
    Read Conrad and Barker (2010)

    \begin{itemize}
    \tightlist
    \item
      Conrad, Peter and Kristin K. Barker. 2010. ``The Social Construction of Illness: Key Insights and Policy Implications.'' \emph{Journal of Health and Social Behavior} 51(S):S67-S79. (Link)
    \end{itemize}
  \end{itemize}
\item
  \textbf{After Class:}

  \begin{itemize}
  \tightlist
  \item
    Study for Quiz 1 on Monday, September 27 (Canvas)
  \end{itemize}
\end{itemize}

\newpage

\hypertarget{module-3---culture}{%
\section{Module 3 - Culture}\label{module-3---culture}}

\hypertarget{meeting-3-1---tuesday-september-28---decoding-culture}{%
\subsection*{Meeting 3-1 - Tuesday, September 28 - Decoding Culture}\label{meeting-3-1---tuesday-september-28---decoding-culture}}
\addcontentsline{toc}{subsection}{Meeting 3-1 - Tuesday, September 28 - Decoding Culture}

\begin{itemize}
\tightlist
\item
  \textbf{Before Class:}

  \begin{itemize}
  \tightlist
  \item
    Read \emph{A Sociology Experiment}, ``Chapter 5 - Culture'' (Link)
  \end{itemize}
\end{itemize}

\begin{center}\rule{0.5\linewidth}{0.5pt}\end{center}

\hypertarget{meeting-3-2---thursday-september-30---culture-and-health}{%
\subsection*{Meeting 3-2 - Thursday, September 30 - Culture and Health}\label{meeting-3-2---thursday-september-30---culture-and-health}}
\addcontentsline{toc}{subsection}{Meeting 3-2 - Thursday, September 30 - Culture and Health}

\begin{itemize}
\tightlist
\item
  \textbf{Before Class:}

  \begin{itemize}
  \tightlist
  \item
    Read Acevedo-Garcia and Bates (2008)

    \begin{itemize}
    \tightlist
    \item
      Acevedo-Garcia, Dolores and Lisa M. Bates. 2008. ``Latino Health Paradoxes: Empirical Evidence, Explanations, Future Research, and Implications.'' Pp. 101-113 in \emph{Latinas/os in the United States: Changing the Face of América}, edited by H. Rodríguez, R. Sáenz, and C. Menjívar. New York: Springer. (Canvas)
    \end{itemize}
  \end{itemize}
\item
  \textbf{After Class:}

  \begin{itemize}
  \tightlist
  \item
    Watch \emph{Unnatural Causes}, Part 3 - ``Becoming Americans'' (Pius Library)
  \item
    Complete Theory Isn't Dead 2 for Tuesday, October 5 (Canvas)
  \end{itemize}
\end{itemize}

\newpage

\hypertarget{module-4---socioeconomic-status}{%
\section{Module 4 - Socioeconomic Status}\label{module-4---socioeconomic-status}}

\hypertarget{meeting-4-1---tuesday-october-5---social-class}{%
\subsection*{Meeting 4-1 - Tuesday, October 5 - Social Class}\label{meeting-4-1---tuesday-october-5---social-class}}
\addcontentsline{toc}{subsection}{Meeting 4-1 - Tuesday, October 5 - Social Class}

\begin{itemize}
\tightlist
\item
  \textbf{Before Class:}

  \begin{itemize}
  \tightlist
  \item
    Read \emph{A Sociology Experiment}, ``Chapter 4 - Social Class, Inequality, and Poverty,'' pp.~1-15 (Link)
  \end{itemize}
\end{itemize}

\begin{center}\rule{0.5\linewidth}{0.5pt}\end{center}

\hypertarget{meeting-4-2---thursday-october-7---inequality-and-poverty}{%
\subsection*{Meeting 4-2 - Thursday, October 7 - Inequality and Poverty}\label{meeting-4-2---thursday-october-7---inequality-and-poverty}}
\addcontentsline{toc}{subsection}{Meeting 4-2 - Thursday, October 7 - Inequality and Poverty}

\begin{itemize}
\tightlist
\item
  \textbf{Before Class:}

  \begin{itemize}
  \tightlist
  \item
    Read \emph{A Sociology Experiment}, ``Chapter 4 - Social Class, Inequality, and Poverty,'' pp.~15-35 (Link)
  \end{itemize}
\item
  \textbf{After Class:}

  \begin{itemize}
  \tightlist
  \item
    Complete Sociological Experience 2 for Tuesday, October 12 (Canvas)
  \end{itemize}
\end{itemize}

\begin{center}\rule{0.5\linewidth}{0.5pt}\end{center}

\hypertarget{meeting-4-3---tuesday-october-12---socioeconomic-status-and-health}{%
\subsection*{Meeting 4-3 - Tuesday, October 12 - Socioeconomic Status and Health}\label{meeting-4-3---tuesday-october-12---socioeconomic-status-and-health}}
\addcontentsline{toc}{subsection}{Meeting 4-3 - Tuesday, October 12 - Socioeconomic Status and Health}

\begin{itemize}
\tightlist
\item
  \textbf{Before Class:}

  \begin{itemize}
  \tightlist
  \item
    Read Dow and Rehkopf (2010)

    \begin{itemize}
    \tightlist
    \item
      Dow, William H. and David H. Rehkopf. 2010. ``Socioeconomic gradients in health in international and historical context.'' \emph{Annals of the New York Academy of Sciences} 1186:24-36. (Pius Library)
    \end{itemize}
  \end{itemize}
\item
  \textbf{After Class:}

  \begin{itemize}
  \tightlist
  \item
    Watch \emph{Unnatural Causes}, Part 7 - ``Not Just a Paycheck'' (Pius Library)
  \item
    Make sure you are wrapping up \emph{Mama} Paper 1 for Thursday, October 14 (Canvas)
  \end{itemize}
\end{itemize}

\begin{center}\rule{0.5\linewidth}{0.5pt}\end{center}

\hypertarget{meeting-4-4---thursday-october-14---mama-discussion-1}{%
\subsection*{\texorpdfstring{Meeting 4-4 - Thursday, October 14 - \emph{Mama} Discussion 1}{Meeting 4-4 - Thursday, October 14 - Mama Discussion 1}}\label{meeting-4-4---thursday-october-14---mama-discussion-1}}
\addcontentsline{toc}{subsection}{Meeting 4-4 - Thursday, October 14 - \emph{Mama} Discussion 1}

\begin{itemize}
\tightlist
\item
  \textbf{Before Class:}

  \begin{itemize}
  \tightlist
  \item
    Read \emph{Mama Might Be Better Off Dead}, Chapters 1-5
  \item
    Complete \emph{Mama} Paper 1 (Canvas)
  \end{itemize}
\end{itemize}

\newpage

\hypertarget{module-5---gender-and-sexuality}{%
\section{Module 5 - Gender and Sexuality}\label{module-5---gender-and-sexuality}}

\hypertarget{meeting-5-1---tuesday-october-19---gender-and-health}{%
\subsection*{Meeting 5-1 - Tuesday, October 19 - Gender and Health}\label{meeting-5-1---tuesday-october-19---gender-and-health}}
\addcontentsline{toc}{subsection}{Meeting 5-1 - Tuesday, October 19 - Gender and Health}

\begin{itemize}
\tightlist
\item
  \textbf{Before Class:}

  \begin{itemize}
  \tightlist
  \item
    Read \emph{A Sociology Experiment}, ``Chapter 6 - Gender and Sexuality,'' pp.~1-28 (Link)
  \item
    Read Krieger (2003)

    \begin{itemize}
    \tightlist
    \item
      Krieger, Nancy. 2003. ``Genders, Sexes, and Health: What Are the Connections - and Why Does It Matter?'' \emph{International Journal of Epidemiology} 32(4):652-657. (Canvas)
    \end{itemize}
  \end{itemize}
\end{itemize}

\begin{center}\rule{0.5\linewidth}{0.5pt}\end{center}

\hypertarget{meeting-5-2---thursday-october-21---sexuality-and-health}{%
\subsection*{Meeting 5-2 - Thursday, October 21 - Sexuality and Health}\label{meeting-5-2---thursday-october-21---sexuality-and-health}}
\addcontentsline{toc}{subsection}{Meeting 5-2 - Thursday, October 21 - Sexuality and Health}

\begin{itemize}
\tightlist
\item
  \textbf{Before Class:}

  \begin{itemize}
  \tightlist
  \item
    Read \emph{A Sociology Experiment}, ``Chapter 6 - Gender and Sexuality,'' pp.~28-36 (Link)
  \end{itemize}
\item
  \textbf{After Class:}

  \begin{itemize}
  \tightlist
  \item
    Study for Quiz 2 on Monday, October 25 (Canvas)
  \end{itemize}
\end{itemize}

\newpage

\hypertarget{module-6---race-and-racism}{%
\section{Module 6 - Race and Racism}\label{module-6---race-and-racism}}

\hypertarget{meeting-6-1---tuesday-november-2---race-and-racism-1}{%
\subsection*{Meeting 6-1 - Tuesday, November 2 - Race and Racism 1}\label{meeting-6-1---tuesday-november-2---race-and-racism-1}}
\addcontentsline{toc}{subsection}{Meeting 6-1 - Tuesday, November 2 - Race and Racism 1}

\begin{itemize}
\tightlist
\item
  \textbf{Before Class:}

  \begin{itemize}
  \tightlist
  \item
    Read \emph{A Sociology Experiment}, ``Chapter 7 - Race and Ethnicity,'' pp.~1-23 (Link)
  \item
    Read Cooper et al.~(2003)

    \begin{itemize}
    \tightlist
    \item
      Cooper, Richard S., Jay S. Kaufman, and Ryk Ward. 2003. ``Race and Genomics.'' \emph{New England Journal of Medicine} 348(12):1166-1170. (Link)
    \end{itemize}
  \end{itemize}
\end{itemize}

\begin{center}\rule{0.5\linewidth}{0.5pt}\end{center}

\hypertarget{meeting-6-2--thursday-november-4---race-and-racism-2}{%
\subsection*{Meeting 6-2 -Thursday, November 4 - Race and Racism 2}\label{meeting-6-2--thursday-november-4---race-and-racism-2}}
\addcontentsline{toc}{subsection}{Meeting 6-2 -Thursday, November 4 - Race and Racism 2}

\begin{itemize}
\tightlist
\item
  \textbf{Before Class:}

  \begin{itemize}
  \tightlist
  \item
    \emph{no assignments}
  \end{itemize}
\end{itemize}

\begin{center}\rule{0.5\linewidth}{0.5pt}\end{center}

\hypertarget{meeting-6-3---tuesday-november-9---race-and-racism-3}{%
\subsection*{Meeting 6-3 - Tuesday, November 9 - Race and Racism 3}\label{meeting-6-3---tuesday-november-9---race-and-racism-3}}
\addcontentsline{toc}{subsection}{Meeting 6-3 - Tuesday, November 9 - Race and Racism 3}

\begin{itemize}
\tightlist
\item
  \textbf{Before Class:}

  \begin{itemize}
  \tightlist
  \item
    Read \emph{A Sociology Experiment}, ``Chapter 7 - Race and Ethnicity,'' pp.~23-28 (Link)
  \end{itemize}
\item
  \textbf{After Class:}

  \begin{itemize}
  \tightlist
  \item
    Complete Sociological Experience 3 for Thursday, November 11 (Canvas)
  \end{itemize}
\end{itemize}

\begin{center}\rule{0.5\linewidth}{0.5pt}\end{center}

\hypertarget{meeting-6-4---thursday-november-11---racism-and-health}{%
\subsection*{Meeting 6-4 - Thursday, November 11 - Racism and Health}\label{meeting-6-4---thursday-november-11---racism-and-health}}
\addcontentsline{toc}{subsection}{Meeting 6-4 - Thursday, November 11 - Racism and Health}

\begin{itemize}
\tightlist
\item
  \textbf{Before Class:}

  \begin{itemize}
  \tightlist
  \item
    Read Jones (2006)

    \begin{itemize}
    \tightlist
    \item
      Jones, David S. 2006. ``The persistence of American Indian health disparities.'' \emph{American Journal of Public Health} 96(12): 2122-2134. (Link)
    \end{itemize}
  \item
    Read Sarche and Spicer (2008)

    \begin{itemize}
    \tightlist
    \item
      Sarche, Michelle, and Paul Spicer. 2008. ``Poverty and health disparities for American Indian and Alaska Native children.'' \emph{Annals of the New York Academy of Sciences} 1136(1): 126-136. (Link)
    \end{itemize}
  \item
    Read Williams and Sternthal (2010)

    \begin{itemize}
    \tightlist
    \item
      Williams, David R. and Michelle Sternthal. 2010. ``Understanding Racial-ethnic Disparities in Health : Sociological Contributions.'' \emph{Journal of Health and Social Behavior} 51(S):S15-S27. (Link)
    \end{itemize}
  \end{itemize}
\item
  \textbf{After Class:}

  \begin{itemize}
  \tightlist
  \item
    Watch \emph{Unnatural Causes}, Part 4 - ``Bad Sugar'' (Pius Library)
  \item
    Make sure you are wrapping up \emph{Mama} Paper 2 for Thursday, November 18 (Canvas)
  \end{itemize}
\end{itemize}

\newpage

\hypertarget{module-7---intersectionality}{%
\section{Module 7 - Intersectionality}\label{module-7---intersectionality}}

\hypertarget{meeting-7-1---tuesday-november-16---mama-discussion-2}{%
\subsection*{\texorpdfstring{Meeting 7-1 - Tuesday, November 16 - \emph{Mama} Discussion 2}{Meeting 7-1 - Tuesday, November 16 - Mama Discussion 2}}\label{meeting-7-1---tuesday-november-16---mama-discussion-2}}
\addcontentsline{toc}{subsection}{Meeting 7-1 - Tuesday, November 16 - \emph{Mama} Discussion 2}

\begin{itemize}
\tightlist
\item
  \textbf{Before Class:}

  \begin{itemize}
  \tightlist
  \item
    Read \emph{Mama Might Be Better Off Dead}, Chapters 6-10
  \item
    Complete \emph{Mama} Paper 2 (Canvas)
  \end{itemize}
\end{itemize}

\begin{center}\rule{0.5\linewidth}{0.5pt}\end{center}

\hypertarget{meeting-7-2---thursday-november-18---intersectionality-and-health}{%
\subsection*{Meeting 7-2 - Thursday, November 18 - Intersectionality and Health}\label{meeting-7-2---thursday-november-18---intersectionality-and-health}}
\addcontentsline{toc}{subsection}{Meeting 7-2 - Thursday, November 18 - Intersectionality and Health}

\begin{itemize}
\tightlist
\item
  \textbf{Before Class:}

  \begin{itemize}
  \tightlist
  \item
    Read Collins et al.~(2004)

    \begin{itemize}
    \tightlist
    \item
      Collins, Jr, James W. et al.~2004. ``Very Low Birthweight in African American Infants: The Role of Maternal Exposure to Interpersonal Racial Discrimination.'' \emph{American Journal of Public Health} 94(12):2132-2138. (Link)
    \end{itemize}
  \item
    Read David and Collins (1997)

    \begin{itemize}
    \tightlist
    \item
      David, Richard J. and James W. Collins, Jr.~1997. ``Differing Birth Weight among Infants of U.S.-Born Blacks, African-Born Blacks, and U.S.-Born Whites.'' \emph{The New England Journal of Medicine} 337:1209-1214. (Link)
    \end{itemize}
  \end{itemize}
\item
  \textbf{After Class:}

  \begin{itemize}
  \tightlist
  \item
    Watch \emph{Unnatural Causes}, Part 2 - ``When the Bough Breaks'' (Pius Library)
  \end{itemize}
\end{itemize}

\newpage

\hypertarget{module-8---urban-sociology}{%
\section{Module 8 - Urban Sociology}\label{module-8---urban-sociology}}

\hypertarget{meeting-8-1---tuesday-november-30---urban-america}{%
\subsection*{Meeting 8-1 - Tuesday, November 30 - Urban America}\label{meeting-8-1---tuesday-november-30---urban-america}}
\addcontentsline{toc}{subsection}{Meeting 8-1 - Tuesday, November 30 - Urban America}

\begin{itemize}
\tightlist
\item
  \textbf{Before Class:}

  \begin{itemize}
  \tightlist
  \item
    Read \emph{A Sociology Experiment}, ``Chapter 12 - Urban Sociology'' (Link)
  \item
    Read selections from \emph{Segregation in St.~Louis: Dismantling the Divide} (Link)

    \begin{itemize}
    \tightlist
    \item
      Chapter 1 - ``Segregation at the center'', pp.~4-13
    \item
      Chapter 2 - ``St.~Louis: A city of promise, a history of segregation'', pp.~14-25
    \item
      Chapter 5 - ``Segregation in St.~Louis today'', pp.~64-85
    \end{itemize}
  \end{itemize}
\item
  \textbf{After Class:}

  \begin{itemize}
  \tightlist
  \item
    Complete Sociological Experience 4 for Thursday, December 2 (Canvas)
  \end{itemize}
\end{itemize}

\begin{center}\rule{0.5\linewidth}{0.5pt}\end{center}

\hypertarget{meeting-8-2---thursday-december-2---cities-and-health}{%
\subsection*{Meeting 8-2 - Thursday, December 2 - Cities and Health}\label{meeting-8-2---thursday-december-2---cities-and-health}}
\addcontentsline{toc}{subsection}{Meeting 8-2 - Thursday, December 2 - Cities and Health}

\begin{itemize}
\tightlist
\item
  \textbf{Before Class:}

  \begin{itemize}
  \tightlist
  \item
    Read selections from \emph{For the Sake of All: A report on the health and well-being of African Americans in St.~Louis and why it matters for everyone} (Link)

    \begin{itemize}
    \tightlist
    \item
      Chapter 1 - ``Introduction: Why consider economics, education, and health together?'', pp.~10-15
    \item
      Chapter 3 - ``Place matters: Neighborhood resources and health'', pp.~26-33
    \item
      Chapter 5 - ``A health profile of African Americans in St.~Louis'', pp.~46-67
    \end{itemize}
  \end{itemize}
\item
  \textbf{After Class:}

  \begin{itemize}
  \tightlist
  \item
    Watch \emph{Unnatural Causes}, Part 5 - ``Place Matters'' (Pius Library)
  \item
    Make sure you are wrapping up \emph{Mama} Paper 3 for Tuesday, December 7 (Canvas)
  \end{itemize}
\end{itemize}

\newpage

\hypertarget{module-9---course-conclusion}{%
\section{Module 9 - Course Conclusion}\label{module-9---course-conclusion}}

\hypertarget{meeting-9-1---tuesday-december-7---mama-discussion-3}{%
\subsection*{\texorpdfstring{Meeting 9-1 - Tuesday, December 7 - \emph{Mama} Discussion 3}{Meeting 9-1 - Tuesday, December 7 - Mama Discussion 3}}\label{meeting-9-1---tuesday-december-7---mama-discussion-3}}
\addcontentsline{toc}{subsection}{Meeting 9-1 - Tuesday, December 7 - \emph{Mama} Discussion 3}

\begin{itemize}
\tightlist
\item
  \textbf{Before Class:}

  \begin{itemize}
  \tightlist
  \item
    Read \emph{Mama Might Be Better Off Dead}, Chapters 11-Epilogue
  \item
    Complete \emph{Mama} Paper 3 (Canvas)
  \end{itemize}
\end{itemize}

\begin{center}\rule{0.5\linewidth}{0.5pt}\end{center}

\hypertarget{meeting-9-2---thursday-december-9---course-conclusion}{%
\subsection*{Meeting 9-2 - Thursday, December 9 - Course Conclusion}\label{meeting-9-2---thursday-december-9---course-conclusion}}
\addcontentsline{toc}{subsection}{Meeting 9-2 - Thursday, December 9 - Course Conclusion}

\begin{itemize}
\tightlist
\item
  \textbf{Before Class:}

  \begin{itemize}
  \tightlist
  \item
    Read Quadagno (2010)

    \begin{itemize}
    \tightlist
    \item
      Quadagno, Jill. 2010. ``Institutions, Interest Groups, and Ideology: An Agenda for the Sociology of Health Care Reform.'' \emph{Journal of Health and Social Behavior} 51(2):125-136. (Link)
    \end{itemize}
  \item
    Read Williams (2010)

    \begin{itemize}
    \tightlist
    \item
      Williams, David. 2010. ``Beyond The Affordable Care Act: Achieving Real Improvements In Americans' Health.'' \emph{Health Affairs} 29(8):1481-1488. (Link)
    \end{itemize}
  \end{itemize}
\item
  \textbf{After Class:}

  \begin{itemize}
  \tightlist
  \item
    Study for Quiz 3 on Tuesday, December 14 (Canvas)
  \item
    Complete the Final Project, which is due on Tuesday, December 14 (Canvas)
  \end{itemize}
\end{itemize}

\hypertarget{part-honors-supplement}{%
\part{Honors Supplement}\label{part-honors-supplement}}

\hypertarget{honors-overview}{%
\chapter{Honors Overview}\label{honors-overview}}

This is the supplemental syllabus for the honors section of SOC 1120-05 (i.e.~SOC 1120-H05). Students enrolled in the honors section are expected to complete all \href{/lecture-schedule.html}{readings} and \href{/assignments-and-grading.html}{assignments} included in the syllabus and should look to the previous sections \href{/course-policies.html}{general course policies} as well. The following sections detail the additional work expected of students in the honors section.

\hypertarget{in-depth-seminars}{%
\section{In-Depth Seminars}\label{in-depth-seminars}}

\begin{rmdtip}
Seminar attendance and participation is captured in your participation
grade.
\end{rmdtip}

You are expected to complete three seminars over the course of the semester. The goal of the seminars is to go more in-depth on several topics covered during the class and to provide an additional series of venues to develop your analytic and communication skills. Each seminar will consist of three interrelated parts:

\begin{enumerate}
\def\labelenumi{\arabic{enumi}.}
\tightlist
\item
  in-depth readings,
\item
  a group seminar meeting, and
\item
  a short response paper.
\end{enumerate}

Seminar topics will be selected by the students as a group, and should align with \href{/honors-seminar-topics.html}{the topics listed below}. Consensus on seminar topics should be reached by class on \textbf{Monday, August 31}. All students will complete the same seminars. Once the seminars are selected, Chris will provide a list of additional journal articles and book chapters to read. You will be expected to retrieve readings from SLU's Library if they are available there. If they are not, Chris will provide everyone with a \texttt{pdf} copy of the readings. Approximately 40 to 50 pages of additional reading per seminar should be expected.

We will schedule a when all students are available to meet together and discuss the assigned readings. Each seminar meeting will occur after the lectures on the seminar topic (ideally later that week or the next week). Chris will coordinate scheduling the seminar meetings, which will take place in one of the Sociology Department's conference rooms. You should come to seminar meetings prepared to discuss the readings themselves and draw analytic connections to the course material from both the related week and prior weeks. Attendance and participation in the seminars themselves will be factored into your honors participation grade.

\hypertarget{paper-format}{%
\subsection{Paper Format}\label{paper-format}}

These discussions should inform a response paper written by each student that connects a theme from the in-depth readings and the specific content of the papers to the course as a whole. The paper should be formatted using the following guidelines:

\begin{itemize}
\tightlist
\item
  Times New Roman font
\item
  12 point font size
\item
  Double spaced
\item
  1'' margins on top, bottom, and sides
\item
  Papers should be no less than four and no longer than five pages in length
\item
  Papers should use in-text, parenthetical citations formatted using American Sociological Association (ASA) standards
\item
  Papers should include a works cited section on a separate page that is properly formatted using ASA standards - this does not count towards the page limit
\end{itemize}

\hypertarget{grading-1} of your final grade
\end{rmdtip}

Paper due-dates will be a week after each seminar meeting, and papers will be submitted via Blackboard. Each response paper will be graded on four elements:

\begin{enumerate}
\def\labelenumi{\arabic{enumi}.}
\tightlist
\item
  Content (35 points): How well does the paper synthesize information from the in-depth readings, other course readings, the course lectures, and other course resources such as documentaries (as appropriate).
\item
  Organization (10 points): How well organized is the paper? Does it have an introduction, a conclusion, and a thesis?
\item
  Writing (10 points): How well written is the paper? Is it free of spelling and grammatical errors?
\item
  Citations (5 points): Are citations correctly applied?
\end{enumerate}

Papers are worth approximately 6.1\% of your final grade for each paper.

\hypertarget{op-ed} of your final grade
\end{rmdtip}

You should pick a topic from \href{/honors-seminar-topics.html}{the honors seminar topic list} that you would like to use as a jumping-off point for writing an op-ed style article. With a topic in-hand, you should read ahead and pick \emph{one narrow aspect} of the topic to focus on. Topics must be selected by \textbf{Monday, February 22} and are ``submitted'' by coming to office hours (or making an appointment to see Chris if you have a scheduling conflict) to discuss your idea.

Once you meet with Chris, you should complete additional background research on the topic, and put together a 700 to 800 word op-ed style article. This article should introduce your topic succinctly and \emph{take a position} on it, providing specific data that supports your position. Remember that the main purpose of an op-ed is to persuade your reader. \href{https://styleguide.duke.edu/toolkits/writing-media/how-to-write-an-op-ed-article/}{Duke University} provides some excellent guidance on forming your argument and approaching editorial writing.

Since this is an op-ed, you do not need to provide citations in the body of the paper. Instead, provide end-notes that describe the source of information. Please provide a bibliography on a separate page of your submission that lists all sources of data and information included in the op-ed.

A draft submission, which should contain all required elements, is due on \textbf{Tuesday, April 5} for feedback via Blackboard. It is ungraded; however, if not submitted, a 3\% deduction will be applied to your final grade.

The final submission on \textbf{Monday, May 10} should be made to two venues - Chris via Blackboard and a publication. The publication could be your hometown paper or a local St.~Louis publication if the topic is about St.~Louis. Any publication that accepts op-ed submissions without invitation is acceptable. You don't have to have your op-ed published, just submitted! Make sure to forward your submission email to Chris after it is completed.

\hypertarget{op-ed-format}{%
\subsection{Op-Ed Format}\label{op-ed-format}}

The op-ed should be formatted using the following guidelines:

\begin{itemize}
\tightlist
\item
  Times New Roman font
\item
  12 point font size
\item
  Double spaced
\item
  1'' margins on top, bottom, and sides
\item
  700 to 800 words
\item
  End notes and a bibliography are used to provide sourcing and attribution, and do not count towards the word limit.
\end{itemize}

\hypertarget{op-ed-seminar}{%
\subsection{Op-Ed Seminar}\label{op-ed-seminar}}

Once submitted, op-eds will be shared with all honors students. Each of you should read the other op-eds and come prepared to discuss them during a final seminar session (date TBA). As with the other seminars, attendance and participation will be factored into your honors participation grade.

\hypertarget{grading-2}{%
\subsection{Grading}\label{grading-2}}

Op-eds will be submitted by \textbf{Wednesday, December 14} via Blackboard. Each op-ed paper will be graded on four elements:

\begin{enumerate}
\def\labelenumi{\arabic{enumi}.}
\tightlist
\item
  Content (45 points): How persuasive is the argument? What evidence is used to support the argument?
\item
  Organization (10 points): How well organized is the paper? Does it
  have a clear thesis and present persuasive evidence in a clear, linear manner?
\item
  Writing (15 points): How well written is the paper? Is it free of
  spelling and grammatical errors?
\item
  Bibliography Page (5 points): Are citations correctly applied?
\end{enumerate}

\hypertarget{honors-seminar-topics}{%
\section{Honors Seminar Topics}\label{honors-seminar-topics}}

\begin{table}

\caption{\label{tab:unnamed-chunk-4}SOC 1120 Honors Seminar Topics}
\centering
\begin{tabular}[t]{ll}
\toprule
Module & Topic\\
\midrule
2 & Nature or Nurture?\\
2 & Social Structure\\
2 & Socialization\\
2 & The Institution of Medicine\\
2 & Medicalization\\
\addlinespace
3 & Culture\\
4 & Socioeconomic Status\\
5 & Gender and Sexuality\\
6 & Race and Racism\\
7 & Intersectionality and Health\\
\addlinespace
8 & Urban Sociology\\
\bottomrule
\end{tabular}
\end{table}

\hypertarget{honors-grading}{%
\section{Honors Grading}\label{honors-grading}}

The following point totals supersede the information on the main syllabus under Section 3.4 - ``Grading'' if you are enrolled in the honors section of SOC 1120.

\begin{table}

\caption{\label{tab:unnamed-chunk-5}SOC 1120 Points Breakdown}
\centering
\begin{tabular}[t]{llllll}
\toprule
Assignment & Period & Points & Quantity & Total & Percent\\
\midrule
Participation & Part 1 & 20 pts & x1 & 20 pts & 11.6\%\\
 & Part 2 & 30 pts & x1 & 30 pts & \\
 & Part 3 & 30 pts & x1 & 30 pts & \\
 & Part 4 & 15 pts & x1 & 15 pts & \\
Theory Isn't Dead &  & 20 pts & x2 & 40 pts & 4.9\%\\
\addlinespace
Sociological Experiences &  & 20 pts & x4 & 80 pts & 9.9\%\\
QHQs &  & 75 pts & x2 & 150 pts & 18.4\%\\
Quizzes &  & 50 pts & x3 & 150 pts & 18.4\%\\
Final Project &  & 75 pts & x1 & 75 pts & 9.2\%\\
Honors Response Papers &  & 50 pts & x3 & 150 pts & 18.4\%\\
\addlinespace
Honors Op-Ed &  & 75 pts & x1 & 75 pts & 9.2\%\\
\bottomrule
\end{tabular}
\end{table}

The information listed in Sections 3.4.1 and 3.4.2 still applies to honors students.

\hypertarget{honors-schedule}{%
\chapter{Honors Schedule}\label{honors-schedule}}

\begin{rmdwarning}
Once the honors seminars have been selected by students enrolled in the
honors seminar, this section will be updated with meeting dates, reading
assignments, and response paper due dates.
\end{rmdwarning}

  \bibliography{book.bib,packages.bib}

\end{document}
